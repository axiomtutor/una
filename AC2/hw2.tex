\documentclass{article}
\usepackage{amsmath, amssymb}
\usepackage[dvipsnames]{xcolor}
\usepackage{wasysym}

\DeclareMathAlphabet{\mymathbb}{U}{BOONDOX-ds}{m}{n}

\begin{document}
  \begin{center} \Large
    MA-652 Advanced Calculus\\
    Homework 2, Jan. 13 \\
    Adam Frank
  \end{center}

  \vspace{1cm}

  {\Large \color{Sepia} Problem 1. Rudin page 114 problem 7. Suppose $f'(x),g'(x)$ exist, $g'(x)\ne 0$, and $f(x)=g(x)=0$.  Prove that

  \begin{align*}
    \lim_{t\to x}\frac{f(t)}{g(t)} = \frac{f'(x)}{g'(x)}
  \end{align*} }

  \vspace{1cm}

  \begin{align*}
    \lim_{t\to x}\frac{f(t)}{g(t)} = \lim_{t\to x}\frac{(f(t)-f(x))/(t-x)}{(g(t)-g(x))/(t-x)} = \frac{\lim_{t\to x}(f(t)-f(x))/(t-x)}{\lim_{t\to x}(g(t)-g(x))/(t-x)} = \frac{f'(x)}{g'(x)}
  \end{align*}

  Where the second equality is due to the fact that $g'(x)\ne 0$.

  \pagebreak

  {\Large \color{Sepia} Problem 2. Rudin page 114 problem 8. Suppose $f'$ is continuous on $[a,b]$ and $\varepsilon>0$.  Prove that there exists a $\delta>0$ such that $\left|\frac{f(t)-f(x)}{t-x}-f'(x)\right|<\varepsilon$ whenever $0<|t-x|<\delta$, for $a\leq x\leq b$, $a\leq t\leq b$.  Does this hold for vector-valued functions?}

  \vspace{1cm}

  Since $f'$ is continuous on a compact set it is uniformly continuous.  Hence

  \begin{align*}
    |f'(t)-f'(x)|<\varepsilon
  \end{align*}

  for some $\delta_1$ and all $|t-x|<\delta_1$.  Hence

  \begin{align*}
    \left|f'(t)+\frac{f(x)-f(t)}{x-t}-\frac{f(x)-f(t)}{x-t}-f'(x)\right| = 
  \end{align*}

  \pagebreak

  {\Large \color{Sepia} Problem 3. Rudin page 115 problem 9. Let $f$ be a continuous real function on $\mathbb R$, of which it is known that $f'(x)$ exists for all $x\ne 0$ and that $f'(x)\to 3$ as $x\to 0$.  Does it follow that $f'(0)$ exists?}

  \vspace{1cm}

  \pagebreak

  {\Large \color{Sepia} Problem 4. Rudin page 115 problem 11. Suppose $f$ is defined in a neighborhood of $x$, and suppose $f'(x)$ exists.  Show that

  \begin{align*}
    \lim_{h\to 0} \frac{f(x+h)+f(x-h)-2f(x)}{h^2} = f''(x)
  \end{align*}

  Show by an example that the limit may exist even if $f''(x)$ does not. {\it Hint: Use Theorem 5.13.}}

  \vspace{1cm}

  \pagebreak

  {\Large \color{Sepia} Problem 5. Rudin page 115 problem 12. If $f(x)=|x^3|$, compute $f'(x), f''(x)$ for all real $x$, and show that $f^{(3)}(0)$ does not exist.}

  \vspace{1cm}

  \pagebreak

  {\Large \color{Sepia} Problem 6. Let $a\in \mathbb R$ and $f:(a,\infty)\to \mathbb R$ be twice differentiable. (a) Use Taylor's Theorem to show that $f'(x)=\frac{1}{2h}[f(x+2h)-f(x)]-hf''(\xi)$ for some $\xi \in (x,x+2h)$.}

  \vspace{1cm}

  \vspace{1cm}

  {\Large \color{Sepia} (b) Use the result from part (a) to show that if $M_0,M_1,M_2$ are the LUBs of $|f(x)|,|f'(x)|,$ and $|f''(x)|$ respectively on $(a,\infty)$, then $|f'(x)|\leq hM_2+\frac{M_0}{h}$.}

  \vspace{1cm}

  \vspace{1cm}

  {\Large \color{Sepia} (c) Use part (b) tos how $M_1^2\leq 4M_0M_2$.}

  \vspace{1cm}

  \pagebreak

  {\Large \color{Sepia} Problem 7. Rudin part 116 problem 16. Suppose $f$ is twice differentiable on $(0,\infty)$, $f''$ is bounded on $(0,\infty)$, and $f(x)\to 0$ as $x\to \infty$.  Prove that $f'(x)\to 0$ as $x\to \infty$.  {\it Hint:} Let $a\to \infty$ in exercise 15.}

  \vspace{1cm}

  \pagebreak

  {\Large \color{Sepia} Problem 8. Rudin part 116 problem 25. Suppose $f$ is twice differentiable on $[a,b], f(a)<0, f(b)>0, f'(x)\geq \delta>0$, and $0\leq f''(x)\leq M$ for all $x\in[a,b]$.  Let $\xi$ be the unique point in $(a,b)$ at which $f(\xi)=0.$  Complete the details in the following outline of Newton's method for computing $\xi$.

  (a) Choose $x_1\in (\xi,b)$ and define $\{x_n\}$ by

  \begin{align*}
    x_{n+1}=x_n-\frac{f(x_n)}{f'(x_n)}
  \end{align*}

  Interpret this geometrically in terms of the tangent to the graph of $f$.}

  \vspace{1cm}

  \vspace{1cm}

  {\Large \color{Sepia} (b) Prove that $x_{n+1}<x_n$ and that $\lim_{n\to \infty}x_n=\xi$.}

  \vspace{1cm}

  \vspace{1cm}

  {\Large \color{Sepia} (c) Use Taylor's theorem to show that $x_{n+1}-\xi = \frac{f''(t_n)}{2f'(x_n)}(x_n-\xi)^2$ for some $t_n\in(\xi,x_n)$.}

  \vspace{1cm}

  \vspace{1cm}

  {\Large \color{Sepia} (d) If $A=M/2\delta$ deduce that

  \begin{align*}
    0\leq x_{n+1}-\xi \leq \frac 1 A [A(x_1-\xi)]^{2n}
  \end{align*}}

  \vspace{1cm}

  \vspace{1cm}

  {\Large \color{Sepia} (e) Show that Newton's method amounts to finding a fixed point of the function $g$ defined by

  \begin{align*}
    g(x) = x-\frac{f(x)}{f'(x)}
  \end{align*}

  How does $g'(x)$ behave for $x$ near $\xi$?}

  \vspace{1cm}

  \vspace{1cm}

  {\Large \color{Sepia} (f) Put $f(x)=x^{1/3}$ on $(-\infty,\infty)$ and try Newton's method.  What happens?}

  \vspace{1cm}

  \pagebreak



































\end{document}
