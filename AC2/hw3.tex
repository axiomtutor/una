\documentclass{article}
\usepackage{amsmath, amssymb}
\usepackage[dvipsnames]{xcolor}
\usepackage{wasysym}
\usepackage{graphicx}
\usepackage{mathrsfs}

\DeclareMathAlphabet{\mymathbb}{U}{BOONDOX-ds}{m}{n}

\begin{document}
  \begin{center} \Large
    MA-652 Advanced Calculus\\
    Homework 3, Jan. 26 \\
    Adam Frank
  \end{center}

  \vspace{1cm}

  {\Large \color{Sepia} Problem 1. Let $f(x)=2x+1$ over the interval $[1,3]$ and let $P$ be the partition consisting of the points $\{1,\frac 3 2, 2, 3 \}$.

  (a.) Compute $L(P,f),U(P,f)$, and $U(P,f)-L(P,f)$.}

  \vspace{1cm}

  Using the notation common in the text, where $m_i$ refers to the infimum over a given interval in the partition, and $M_i$ refers to the supremum,

  \begin{align*}
    \Delta x_1 = \frac 1 2, \ \Delta x_2 = \frac 1 2, \ \Delta x_3 = 1\\\\
    m_1 = 3, \ m_2 = 4, \ m_3 = 5\\\\
    M_1 = 4, \ M_2 = 5, \ M_3 = 7
  \end{align*}

  Therefore

  \begin{align*}
    L(P,f)&=(3)\frac 1 2 + (4) \frac 1 2 + (5)\cdot 1 = \frac{17}{2}\\\\
    U(P,f)&=(4)\frac 1 2 + (5)\frac 1 2 + 7\cdot 1 = \frac{23}{2}\\\\
    U(P,f)-L(P,f)&=\frac{6}{2}=3
  \end{align*}



  \vspace{1cm}

  {\Large \color{Sepia} (b.) What happens to $U(P,f)-L(P,f)$ when the point $\frac 5 2$ is added to the partition?}

  \vspace{1cm}

  The computations on the first two intervals are unchanged.  $\Delta x_3$ is replaced by $\Delta x_3=\frac 1 2$, $m_3=5$, and $M_3 = 6$.  Also we now have $\Delta x_4 = \frac 1 2$, $m_4=6$, and $M_4=7$.  So if we re-use the previous computation, the new difference is

  \begin{align*}
    U(P,f)-L(P,f)-\overbrace{[(7)\cdot 1-(5)\cdot 1]}^{\text{remove $(M_3-m_3)\Delta x_3$}} + \overbrace{[6-5]\frac 1 2 + [7-6]\frac 1 2}^{\text{insert new $(M_3-m_3)\Delta x_3 +(M_4-m_4)\Delta x_4$}} \\\\
    = 3-2+1 = 2
  \end{align*}

  \vspace{1cm}

  {\Large \color{Sepia} (c.) Find a partition $P'$ of $[1,3]$ for which $U(P',f)-L(P',f)<2$.}

  \vspace{1cm}

  Using $P'=\{1,1.25,1.5,2,2.5,3\}$ if we call the partition from art (b.), $P^*=\{1,1.5,2,2.5,3\}$ then

  \begin{align*}
    U(P',f)-L(P',f) = \\
    U(P^*,f)-L(P^*,f)-(4-3)\frac 1 2+\left([3.5-3][.25]+[4-3.5][.25]\right) = \\
    2 -\frac 1 2 + \frac 1 8 +\frac 1 8 = 2-\frac 1 4 = 1.75
  \end{align*}

  \pagebreak


  {\Large \color{Sepia} Problem 2. Show that the function $f(x)=k$ where $k\in\mathbb R$ is integrable and find $\int_a^b f\ d\alpha$. }

  \vspace{1cm}

  We can show this directly since, if $P=\{x_0,\dots,x_n\}$ is any partition,

  \begin{align*}
    L(P,f)=\sum_{i=1}^n \inf(f)\Delta \alpha(x_i) = \sum_{k=1}^n k\Delta \alpha(x_i) = k\sum_{i=1}^n \Delta \alpha(x_i) \\
    = k\cdot (\alpha(x_n)-\alpha(x_1)) = k\cdot (\alpha(b)-\alpha(a))
  \end{align*}

  The above depends on the fact that $\sum \Delta \alpha(x_i) = \alpha(x_n)-\alpha(x_0)$.  This holds because the sum is telescoping, and the middle terms cancel leaving only the first and last term.

  Since the above is true for each partition $P$ then $\underline \int_{a}^b f\ d\alpha = k\cdot (\alpha(b)-\alpha(a))$.  By the same argument we also have $U(P,f)=k(\alpha(b)-\alpha(a))$ and therefore $\overline \int_a^b f\ d\alpha$.  Therefore

  \begin{align*}
    \underline \int_a^b f\ d\alpha = \overline \int_a^b f \ d\alpha = \int_a^b f \ d\alpha
  \end{align*}

  \pagebreak
  {\Large \color{Sepia} Problem 3. Rudin page 138 problem 1. Suppose $\alpha$ increases on $[a,b]$, $a\leq x_0\leq b$, $\alpha$ is continuous at $x_0$, $f(x_0)=1$, and $f(x)=0$ if $x\ne x_0$.  Prove that $f\in\mathscr R(\alpha)$ and that $\int f \ d\alpha=0$.  }

  \vspace{1cm}

  First we show that $L(P,f,\alpha)=0$ for any partition $P$ of $[a,b]$.  If $P=\{y_0,\dots,y_n\}$ and if $0 < i \leq n$ then the infimum on $\displaystyle m_i=\inf_{y_{i-1}\leq y \leq y_i}f(y) = 0$.  Therefore each term of $L(P,f,\alpha)=\displaystyle\sum_{i=1}^n m_i\Delta\alpha_i$ is 0 and therefore $L(P,f,\alpha)=0$.  Hence $\underline\int_a^b f \ d\alpha = \sup\{0\}=0$.

  Next we show that for any $\varepsilon\in\mathbb R^+$ we have $\overline\int_a^b f \ d\alpha < \varepsilon$.  In particular we do this by showing that for each $\varepsilon$ there exists a partition $P$ such that $U(P,f,\alpha)<\varepsilon$.  Now since $\alpha$ is continuous at $x_0$ there exists a $\delta\in\mathbb R^+$ such that if $|y-x_0|<\delta$ then $|\alpha(y)-\alpha(x_0)|<\varepsilon/2$.  Set

  \begin{align*}
    y_1 &= \max\{a, x_0-\delta\}\\
    y_2 &= \min\{x_0+\delta, b\}
  \end{align*}

  so that for all $y_1<y<y_2$ we have that $|y-x_0|<\delta$ and therefore $|\alpha(y)-\alpha(x_0)|<\varepsilon$.  In particular note that this implies, since $\alpha$ is monotonically increasing,

  \begin{align*}
    \alpha(y_2)-\alpha(y_1) = |\alpha(y_2)-\alpha(y_1)| &\leq |\alpha(y_2)-\alpha(x_0)|+|\alpha(y_1)-\alpha(x_0)| \\\\
    &< \varepsilon/2 +\varepsilon/2 = \varepsilon
  \end{align*}

  Now form the partition $P=\{y_0=a,y_1,y_2,y_3=b\}$ so that the suprema are $M_1=0,M_2=1,M_3=0$.  Hence

  \begin{align*}
    U(P,f,\alpha) = 0+\Delta\alpha_2+0=\alpha(y_2)-\alpha(y_1)<\varepsilon
  \end{align*}

  Since $\overline\int_a^b f \ d\alpha \leq U(P,f,\alpha)<\varepsilon$, this shows that $\overline\int_a^b f \ d\alpha =0$.  We now have that

  \begin{align*}
    \underline\int_a^b f\ d\alpha = \overline \int_a^b f \ d\alpha =0
  \end{align*}

  which shows both $f\in\mathscr R(\alpha)$ and $\int f \ d\alpha = 0$.

  \pagebreak
  {\Large \color{Sepia} Problem 4. Rudin page 138 problem 2. Suppose $f\geq 0$, $f$ is continuous on $[a,b]$, and $\int_a^b f(x)\ dx = 0$.  Prove that $f(x)=0$ for all $x\in [a,b]$.}

  \vspace{1cm}

  Suppose for contradiction that $f(x_0)>0$ for some $x_0\in[a,b]$.  Since $f$ is continuous then let $\delta\in\mathbb R^+$ be such that if $|x-x_0|<\delta$ then $|f(x)-f(x_0)| < f(x_0)/2$.  Now set

  \begin{align*}
    y_1 &= \max\{a, x_0-\delta/2\}\\
    y_2 &= \min\{b, x_0+\delta/2\}
  \end{align*}

  It follows that on the interval $[y_1,y_2]$ the least value that $f(x)$ can take is $f(x_0)/2$ and therefore $\displaystyle f(x_0)/2\leq \inf_{y_1<y<y_2} f(y)=m_2$.  If we then call $P=\{y_0=a,y_1,y_2,y_3=b\}$ then we have that

  \begin{align*}
    \int_a^b f \ d\alpha = \underline \int_a^b f\ d\alpha \geq L(P,f) = \sum_{i=1}^3 m_i \Delta y_i
  \end{align*}

  Now since $f\geq 0$ it follows that $m_i\geq 0$ for each $i=1,2,3$, and therefore each term of $L(P,f)$ is non-negative.  Hence

  \begin{align*}
    \sum_{i=1}^3 m_i\Delta y_i \geq m_2\Delta y_2 = m_2\Delta y_2 >0
  \end{align*}

  where the final inequality follows because both $\frac{f(x_0)}{2}$ and $\Delta y_2$ are strictly positive.  But this implies $\int_a^b f\ d\alpha>0$ which contradicts that $\int_a^b f\ dx = 0$.

  \pagebreak
  {\Large \color{Sepia} Problem 5. Let $f\in\mathscr R(\alpha,[a,b])$ and $c\in \mathbb R$.  Prove that if $a<c<b$ then $f\in \mathscr R(\alpha,[a,c])\cap \mathscr R(\alpha,[c,b])$ and $\int_a^b f\ d\alpha = \int_a^c f\ d\alpha+\int_c^b f\ d\alpha$. }

  \vspace{1cm}

  We start by showing that $\int_a^c f\ d\alpha$ exists.  For each $\varepsilon$ we are already gauranteed from the assumptions that there exists a partition $P$ of $[a,b]$ such that $U(P,f,\alpha)-L(P,f,\alpha)<\varepsilon$.  We further refine $P$ with $c$, that is to say, set $P'=P\cup \{c\}$ and then

  \begin{align*}
    U(P',f,\alpha)-L(P',f,\alpha) \leq U(P,f,\alpha)-L(P,f,\alpha) < \varepsilon
  \end{align*}

  Let $P'=\{x_0=a,\dots,x_j=c,\dots,x_n=b\}$, so that in particular $c$ has index $j$ in $P'$.  Since every term of $\displaystyle U(P',f,\alpha)-L(P',f,\alpha) = \sum_{i=1}^n(M_i-m_i)\Delta \alpha_i$ is non-negative then $\displaystyle \sum_{i=1}^{j}(M_i-m_i)\Delta\alpha_i \leq U(P',f,\alpha)-L(P',f,\alpha)<\varepsilon$.  This then shows that for $\varepsilon$ there corresponds the partition $ P'_1 = \{x_0=a, x_1, \dots, x_{j}=c\}$ such that

  \begin{align*}
    U( P'_1, f, \alpha)-L(P'_1, f, \alpha) < \varepsilon
  \end{align*}

  which shows $f\in\mathscr R(\alpha,[a,c])$.  Similarly we define $P'_2 = \{x_j=c,x_{j+1},\dots,x_n\}$.  It follows that

  \begin{align*}
    U(P'_2,f,\alpha)-L(P'_2,f,\alpha) <\varepsilon
  \end{align*}

  and therefore $f\in \mathscr R(\alpha,[c,b])$.

  Now to show that $\int_a^bf\ d\alpha = \int_a^c f \ d\alpha + \int_c^b f\ d\alpha$ we will show

  \begin{align*}
    \left| \int_a^b f\ d\alpha - \left(\int_a^c f\ d\alpha + \int_c^b f\ d\alpha\right)\right|<\varepsilon
  \end{align*}

  Using the notation above, we have

  \begin{align*}
    L(P',f,\alpha,[a,b])\leq \int_a^b f \ d\alpha \leq U(P',f,\alpha,[a,b]) \\\\
    L(P'_1,f,\alpha,[a,c])\leq \int_a^c f \ d\alpha \leq U(P'_1,f,\alpha,[a,c]) \\\\
    L(P'_2,f,\alpha,[c,b])\leq \int_c^b f \ d\alpha \leq U(P'_2,f,\alpha,[c,b]) \\\\
  \end{align*}

  and therefore

  \begin{align*}
    &L(P',f,\alpha,[a,b])-(U(P'_1,f,\alpha,[a,c])+U(P'_2,f,\alpha,[c,b])) \\
    \leq \ &\int_a^b f \ d\alpha - \left(\int_a^c f \ d\alpha +\int_c^b f \ d\alpha\right) \\
    \leq \ & U(P',f,\alpha,[a,b])-(L(P'_1,f,\alpha,[a,c])+L(P'_2,f,\alpha,[c,b]))
  \end{align*}

  But from the earlier work, we already have that

  \begin{align*}
    &U(P',f,\alpha,[a,b])-(L(P'_1,f,\alpha,[a,c])+L(P'_2,f,\alpha,[c,b])) \\\\
    = \ &U(P',f,\alpha,[a,b])-L(P',f,\alpha,[a,b]) < \varepsilon
  \end{align*}

  and therefore also

  \begin{align*}
    -\varepsilon < &L(P',f,\alpha,[a,b])-U(P',f,\alpha,[a,b]) \\\\
    &= L(P',f,\alpha,[a,b])-(U(P'_1,f,\alpha,[a,c])+U(P'_2,f,\alpha,[c,b]))
  \end{align*}

  Hence we have

  \begin{align*}
    -\varepsilon < \int_a^b f \ d\alpha - \left(\int_a^c f \ d\alpha +\int_c^b f \ d\alpha\right)<\varepsilon
  \end{align*}

  which entails $\int_a^b f \ d\alpha - \left(\int_a^c f \ d\alpha + \int_c^b f \ d\alpha\right)=0$.  Therefore

  \begin{align*}
    \int_a^b f \ d\alpha = \int_a^c f \ d\alpha + \int_c^b f \ d\alpha
  \end{align*}

  \pagebreak
  {\Large \color{Sepia} Problem 6. Prove that if $f\in\mathscr R(\alpha_1,[a,b])\cap \mathscr R(\alpha_2,[a,b])$ then $f\in \mathscr R(\alpha_1+\alpha_2,[a,b])$ and $\int_a^bf\ d(\alpha_1+\alpha_2) = \int_a^bf\ d\alpha_1+\int_a^b f\ d\alpha_2$. }

  \vspace{1cm}

  First notice that if $P=\{x_0=a,\dots,x_n=b\}$ is any partition then

  \begin{align*}
    U(P,f,\alpha_1+\alpha_2)&=\sum_{i=1}^n M_i\Delta(\alpha_1+\alpha_2)_i \\\\
    &= \sum_{i=1}^nM_i\left((\alpha_1(x_{i})+\alpha_2(x_{i}))-(\alpha_1(x_{i-1})+\alpha_2(x_{i-1})\right)\\\\
    &= \sum_{i=1}^n M_i\left((\alpha_1(x_{i})-\alpha_1(x_{i-1})+(\alpha_2(x_{i})-\alpha_2(x_{i-1}))\right) \\\\
    &= \sum_{i=1}^n M_i\Delta(\alpha_1)_i+\sum_{i=1}^nM_i\Delta(\alpha_2)_i \\\\
    &= U(P,f,\alpha_1)+U(P,f,\alpha_2)
  \end{align*}

  The same argument, with $m_i$ replacing $M_i$, shows that $L(P,f,\alpha_1+\alpha_2)=L(P,f,\alpha_1)+L(P,f,\alpha_2)$.

  Now let $\varepsilon\in\mathbb R^+$.  Let $P_1,P_2$ be any partitions of $[a,b]$ such that

  \begin{align*}
    U(P_1,f,\alpha_1) - L(P_1,f,\alpha_1) < \varepsilon/2\\
    U(P_2,f,\alpha_2) - L(P_2,f,\alpha_2) < \varepsilon/2
  \end{align*}

  Then set $P=P_1\cup P_2 = \{x_0=a,\dots,x_n=b\}$ so that

  \begin{align*}
    U(P,f,\alpha_1) - L(P,f,\alpha_1) < \varepsilon/2\\
    U(P,f,\alpha_2) - L(P,f,\alpha_2) < \varepsilon/2
  \end{align*}

  Then summing the above,

  \begin{align*}
    U(P,f,\alpha_1)+U(P,f,\alpha_2)-L(P,f,\alpha_1)-L(P,f,\alpha_2) &< \varepsilon \quad \Rightarrow \\\\
    U(P,f,\alpha_1+\alpha_2)-L(P,f,\alpha_1+\alpha_2) &< \varepsilon
  \end{align*}

  The second inequation above follows from the first, using the facts that we demonstrated at the start.  Hence we have that $f\in\mathscr R(\alpha_1+\alpha_2,[a,b])$.

  Next we show

  \begin{align*}
    \left|\int_a^b f\ d(\alpha_1+\alpha_2) - \left(\int_a^b f \ d\alpha_1+\int_a^b f \ d\alpha_2\right)\right| <\varepsilon
  \end{align*}

  We have the following inequations,

  \begin{align*}
    &L(P, f, \alpha_1+\alpha_2) &&\leq &&\int_a^b f \ d(\alpha_1+\alpha_2) &&\leq &&U(P,f,\alpha_1+\alpha_2)\\
    &L(P, f, \alpha_1) &&\leq &&\int_a^b f \ d\alpha_1 &&\leq &&U(P,f,\alpha_1)\\
    &L(P, f, \alpha_2) &&\leq &&\int_a^b f \ d\alpha_2 &&\leq &&U(P,f,\alpha_2)
  \end{align*}

  Therefore by subtracting the last two from the first,

  \begin{align*}
    &L(P, f, \alpha_1+\alpha_2)-(U(P, f, \alpha_1)+U(P, f, \alpha_2))\\
    \leq \ &\int_a^b f\ d(\alpha_1+\alpha_2) - \left(\int_a^b f \ d\alpha_1+\int_a^b f \ d\alpha_2\right)\\
    \leq \ & U(P, f, \alpha_1+\alpha_2)-(L(P, f, \alpha_1)+L(P, f, \alpha_2))
  \end{align*}

  Now using the principle we showed at the start, the above is the same as

  \begin{align*}
    &L(P, f, \alpha_1+\alpha_2)-U(P, f, \alpha_1+ \alpha_2)\\
    \leq \ &\int_a^b f\ d(\alpha_1+\alpha_2) - \left(\int_a^b f \ d\alpha_1+\int_a^b f \ d\alpha_2\right)\\
    \leq \ & U(P, f, \alpha_1+\alpha_2)-L(P, f, \alpha_1+\alpha_2)
  \end{align*}

  Since we already have

  \begin{align*}
    U(P, f, \alpha_1+\alpha_2)-&L(P, f, \alpha_1+\alpha_2) < \varepsilon\\
    &\text{and so}\\
    -\varepsilon < L(P, f, \alpha_1+\alpha_2)-&U(P, f, \alpha_1+\alpha_2)
  \end{align*}

  then we can infer

  \begin{align*}
    -\varepsilon <  \int_a^b f\ d(\alpha_1+\alpha_2) - \left(\int_a^b f \ d\alpha_1+\int_a^b f \ d\alpha_2\right) < \varepsilon
  \end{align*}

  Hence

  \begin{align*}
    \int_a^b f\ d(\alpha_1+\alpha_2) = \int_a^b f \ d\alpha_1+\int_a^b f \ d\alpha_2
  \end{align*}

  \pagebreak
  {\Large \color{Sepia} Problem 7. Prove that if $f\in\mathscr R(\alpha_1,[a,b])$ and $c\in \mathbb R^+$ then $f\in \mathscr R(c\alpha,[a,b])$ and $\int_a^bf\ d(c\alpha) = c\int_a^bf\ d\alpha$. }

  \vspace{1cm}

  First note that for any partition $P=\{x_0=a,\dots,x_n=b\}$ we have that

  \begin{align*}
    U(P,f,c\alpha)&=\sum_{i=1}^n M_i \Delta(c\alpha)_i \\\\
    &= \sum_{i=1}^n M_i (c\alpha(x_i)-c\alpha(x_{i-1})) \\\\
    &= c\sum_{i=1}^n M_i (\alpha(x_i)-\alpha(x_{i-1})) \\\\
    &= cU(P,f,\alpha)
  \end{align*}

  Therefore

  \begin{align*}
    \overline \int_a^b f \ d(c\alpha) &=\inf\{U(P,f,c\alpha)|P \text{ is a partition of } [a,b]\} \\\\ &= \inf\{cU(P,f,\alpha)|P \text{ is a partition of } [a,b]\} \\\\
    &= c\inf\{U(P,f,\alpha)|P \text{ is a partition of } [a,b]\} \\\\
    &= c\int_a^b f\ d\alpha
  \end{align*}

  The third equation above is due to the fact that $c\in\mathbb R^+$.

  By the same logic as above $\underline \int_a^b f \ d(c\alpha) = c\int_a^b f \ d\alpha$ and therefore

  \begin{align*}
    \int_a^b f \ d(c\alpha)=\overline \int_a^b f \ d(c\alpha)=\underline \int_a^b f \ d(c\alpha)=c\int_a^b f \ d\alpha
  \end{align*}

  \pagebreak
  {\Large \color{Sepia} Problem 8. Rudin page 138 problem 3. Define three functions $\beta_1,\beta_2,\beta_3$ as follows.  $\beta_j(x)=0$ if $x<0$, $\beta_j(x)=1$ if $x>0$ for $j=1,2,3$.  And $\beta_1(0)=0, \beta_2(0)=1,\beta_3(0)=\frac 1 2$.  Let $f$ be a bounded function on $[-1,1]$.

  (a) Prove that $f\in\mathscr R(\beta_1)$ if and only if $f(0+)=f(0)$ and that then

  \begin{align*}
    \int_{-1}^1 f\ d\beta_1 = f(0)
  \end{align*}

  }

  \vspace{1cm}

  Suppose that $f\in\mathscr R(\alpha)$ and let $P'$ be a partition $P'=\{x_0=-1,x_1,\dots,x_n=1\}$ such that

  \begin{align*}
    U(P',f,\beta_1)-L(P',f,\beta_1)<\varepsilon
  \end{align*}

  Now refine $P'$ with $0$, that is to say, set $P=P'\cup\{0\}$.  Then

  \begin{align*}
    U(P,f,\beta_1)-L(P,f,\beta_1)\leq U(P',f,\beta_1)-L(P',f,\beta_1)<\varepsilon
  \end{align*}

  If we say that $0$ has index $j$ in $P$, then $P=\{x_0=-1,\dots,x_j=0,\dots,x_n=1\}$.  Therefore, for each $i < j+1$ we have $\Delta(\beta_1)_i = 0-0=0$ and also for each $i>j+1$ we have $\Delta(\beta_1)_i = 1-1 = 0$.  Moreover, $\Delta(\beta_1)_{j+1}=1-0=1$.  Therefore

  \begin{align*}
    U(P,f,\beta_1)&=0+M_{j+1}+0=M_{j+1}\\
    L(P,f,\beta_1)&=0+m_{j+1}+0=m_{j+1}\\\\
    U(P,f,\beta_1)-L(P,f,\beta_1) &= M_{j+1}-m_{j+1}<\varepsilon
  \end{align*}

  First note that since $f$ is bounded then $f(0+)$ exists.  Now since $f(0),f(0+) \in [m_{j+1},M_{j+1}]$ then $|f(0)-f(0+)|\leq M_{j+1}-m_{j+1} = U(P,f,\beta_1)-L(P,f,\beta_1)< \varepsilon$.  Since $\varepsilon\in\mathbb R^+$ was arbitrary then this implies $|f(0)-f(0+)|=0$ and so $f(0)=f(0+)$.  

  \vspace{1cm}

  From here we can show that $\int_{-1}^1 f \ d\beta_1 = f(0)$ by observing that

  \begin{align*}
    L(P,f,\beta_1)=m_{j+1}&\leq \int_{-1}^1 f \ d\beta_1 \leq M_{j+1}=U(P,f,\beta_1)\\
    m_{j+1}&\leq f(0)\leq M_{j+1}
  \end{align*}

  Subtracting these equations,

  \begin{align*}
    0\leq \int_{-1}^1 f \ d\beta_1 - f(0)\leq 0
  \end{align*}

  so that therefore $\int_{-1}^1 f \ d\beta_1-f(0)=0$ and hence

  \begin{align*}
    \int_{-1}^1 f \ d\beta_1=f(0)
  \end{align*}

  \vspace{1cm}

  Now for the converse, suppose $f(0+)=f(0)$ and let $\varepsilon\in\mathbb R^+$.  Let $\delta\in\mathbb R^+$ be such that if $x\in(0,\delta)$ then $|f(x)-f(0)|<\varepsilon$.  Now set $P=\{x_0=-1,x_1=0,x_2=\delta,x_3=1\}$.  Then

  \begin{align*}
    \Delta(\beta_1)_1 &= 0-0=0 \\
    \Delta(\beta_1)_2 &= 1-0=1 \\
    \Delta(\beta_1)_3 &= 1-1=0
  \end{align*}

  Also notice that

  \begin{align*}
    m_2&=\displaystyle\inf_{[0,\delta]}f\geq f(0)-\varepsilon\\
    M_2&=\displaystyle\sup_{[0,\delta]}f\leq f(0)+\varepsilon\\
  \end{align*}

  therefore

  \begin{align*}
    M_2-m_2 \leq 2\varepsilon
  \end{align*}

  Now we can calculate

  \begin{align*}
    U(P,f,\beta_1)-L(P,f,\beta_1) = M_2-m_2\leq 2\varepsilon
  \end{align*}

  and since $\varepsilon$ was arbitrary, this implies $U(P,f,\beta_1)-L(P,f,\beta_1)$ goes to 0, hence $f\in\mathscr R(\beta_1)$.

  \vspace{1cm}

  Finally, notice that the above entails that if $f(0+)=f(0)$ then $\int_{-1}^1 f\ d\beta_1 = f(0)$.  For if $f(0+)=f(0)$ then by the above, $f\in\mathscr R(\beta_1)$ and then by the first part of this proof it follows that $\int_{-1}^1 f\ d\beta_1=f(0)$.

  \vspace{1cm}

  {\Large \color{Sepia} (b.) State and prove a similar result for $\beta_2$.}

  \vspace{1cm}

  {\it Claim:}  $f\in\mathscr R(\beta_2)$ if and only if $f(0-)=f(0)$.  In that case $\int_{-1}^1 f \ d\beta_2 = f(0)$.

  \vspace{1cm}

  {\it Proof:} Suppose that $f\in\mathscr R(\beta_2)$ and let $P'$ be a partition $P'=\{x_0=a,x_1,\dots,x_n=b\}$ such that

  \begin{align*}
    U(P',f,\beta_2)-L(P',f,\beta_2)<\varepsilon
  \end{align*}

  Now refine $P'$ with $0$, that is to say, set $P=P'\cup\{0\}$.  Then

  \begin{align*}
    U(P,f,\beta_2)-L(P,f,\beta_2)\leq U(P',f,\beta_2)-L(P',f,\beta_2)<\varepsilon
  \end{align*}

  If we say that $0$ has index $j$ in $P$, then $P=\{x_0,\dots,x_j=0,\dots,x_n\}$.  Therefore, for each $i < j$ we have $\Delta(\beta_2)_i = 0-0=0$ and also for each $i>j$ we have $\Delta(\beta_2)_i = 1-1 = 0$.  Moreover, $\Delta(\beta_2)_{j}=1-0=1$.  Therefore

  \begin{align*}
    U(P,f,\beta_2)&=0+M_{j}+0=M_{j}\\
    L(P,f,\beta_2)&=0+m_{j}+0=m_{j}\\\\
    U(P,f,\beta_2)-L(P,f,\beta_2) &= M_{j}-m_{j}<\varepsilon
  \end{align*}

  Now since $f(0),f(0+) \in [x_{j-1},0]$ then $|f(0)-f(0+)|\leq M_{j}-m_j = U(P,f,\beta_1)-L(P,f,\beta_1)< \varepsilon$.  Since $\varepsilon\in\mathbb R^+$ was arbitrary then this implies $|f(0)-f(0+)|=0$ and so $f(0)=f(0+)$.  

  \vspace{1cm}

  From here we can show that $\int_{-1}^1 f \ d\beta_2 = f(0)$ by observing that

  \begin{align*}
    L(P,f,\beta_2)=m_{j}&\leq \int_{-1}^1 f \ d\beta_2 \leq M_{j}=U(P,f,\beta_2)\\
    m_{j}&\leq f(0)\leq M_{j}
  \end{align*}

  Subtracting these equations,

  \begin{align*}
    0\leq \int_{-1}^1 f \ d\beta_2 - f(0)\leq 0
  \end{align*}

  so that therefore $\int_{-1}^1 f \ d\beta_2-f(0)=0$ and hence

  \begin{align*}
    \int_{-1}^1 f \ d\beta_2=f(0)
  \end{align*}

  \vspace{1cm}

  Now for the converse, suppose $f(0-)=f(0)$ and let $\varepsilon\in\mathbb R^+$.  Let $\delta\in\mathbb R^+$ be such that if $x\in(-\delta,0)$ then $|f(x)-f(0)|<\varepsilon$.  Now set $P=\{x_0=-1,x_1=-\delta,x_2=0,x_3=1\}$.  Then

  \begin{align*}
    \Delta(\beta_2)_1 &= 0-0=0 \\
    \Delta(\beta_2)_2 &= 1-0=1 \\
    \Delta(\beta_2)_3 &= 1-1=0
  \end{align*}

  Also notice that

  \begin{align*}
    m_2&=\displaystyle\inf_{[-\delta,0]}f\geq f(0)-\varepsilon\\
    M_2&=\displaystyle\sup_{[-\delta,0]}f\leq f(0)+\varepsilon\\
  \end{align*}

  therefore

  \begin{align*}
    M_2-m_2 \leq 2\varepsilon
  \end{align*}

  Now we can calculate

  \begin{align*}
    U(P,f,\beta_2)-L(P,f,\beta_2) = M_2-m_1\leq 2\varepsilon
  \end{align*}

  and since $\varepsilon$ was arbitrary, this implies $U(P,f,\beta_2)-L(P,f,\beta_2)$ goes to 0, hence $f\in\mathscr R(\beta_2)$.

  \vspace{1cm}

  Finally, notice that the above entails that if $f(0-)=f(0)$ then $\int_{-1}^1 f\ d\beta_2 = f(0)$.  For if $f(0-)=f(0)$ then by the above, $f\in\mathscr R(\beta_2)$ and then by the first part of this proof it follows that $\int_{-1}^1 f\ d\beta_2=f(0)$.

  \vspace{1cm}

  {\Large \color{Sepia} (c.) Prove that $f\in\mathscr R(\beta_3)$ if and only if $f$ is continuous at 0.  }

  \vspace{1cm}

  Suppose that $f\in\mathscr R(\beta_3)$ and let $P'$ be a partition $P'=\{x_0=-1,x_1,\dots,x_n=1\}$ such that

  \begin{align*}
    U(P',f,\beta_3)-L(P',f,\beta_3)<\varepsilon
  \end{align*}

  Now refine $P'$ with $0$, that is to say, set $P=P'\cup\{0\}$.  Then

  \begin{align*}
    U(P,f,\beta_3)-L(P,f,\beta_3)\leq U(P',f,\beta_3)-L(P',f,\beta_2)<\varepsilon
  \end{align*}

  If we say that $0$ has index $j$ in $P$, then $P=\{x_0,\dots,x_j=0,\dots,x_n\}$.  Therefore, for each $i < j$ we have $\Delta(\beta_3)_i = 0-0=0$ and also for each $i>j+1$ we have $\Delta(\beta_3)_i = 1-1 = 0$.  Moreover, $\Delta(\beta_3)_{j}=1/2-0=1/2$ and $\Delta(\beta_3)_{j+1}=1-1/2=1/2$.  Therefore

  \begin{align*}
    U(P,f,\beta_3)&=0+\frac{1}{2}\cdot M_{j}+\frac 1 2 M_{j+1}+0=\frac{M_j+M_{j+1}}{2}\\
    L(P,f,\beta_3)&=0+\frac 1 2 \cdot m_{j}+\frac 1 2 \cdot m_{j+1}+0=\frac{m_{j}+m_{j+1}}{2}\\\\
    U(P,f,\beta_3)-L(P,f,\beta_2) &= \frac{M_{j}+M_{j+1}-m_{j}-m_{j+1}}{2} \\\\
    &= \frac{M_j-m_j}{2}+\frac{M_{j+1}m_{j+1}}{2}<\varepsilon
  \end{align*}

  Now since $\frac{M_j-m_j}{2}\geq 0$ then we have $\frac{M_{j+1}-m_{j+1}}{2}<\varepsilon$ and therefore $\displaystyle \lim_{x\to 0^+}f(x)=f(0)$.  Conversely since $\frac{M_{j+1}-m_{j+1}}{2}\geq 0$ then $\frac{M_{j}-m_{j}}{2} < \varepsilon$ and so $\displaystyle\lim_{x\to 0^-}f(x)=f(0)$.  Hence $f$ is continuous at 0.

  \vspace{1cm}

  Now for the converse, suppose $f$ is continuous at 0.  For a given $\varepsilon\in\mathbb R^+$ set $\delta\in\mathbb R^+$ such that if $|x|<\delta$ then we have $|f(x)-f(0)|<\varepsilon$.  Now define the partition $P=\{x_0=-1,x_1=-\delta, x_2=0,x_3=\delta,x_4=1\}$.  Then

  \begin{align*}
    \Delta(\beta_3)=0, \quad \Delta(\beta_3)=1/2, \quad \Delta(\beta_3)=1/2,\quad \Delta(\beta_3)=0
  \end{align*}

  Also notice that both $m_2, m_3 \geq f(0)-\varepsilon$ and also $M_2,M_3\leq f(0)+\varepsilon$.  Therefore $M_2-m_2, M_3-m_3 \leq 2\varepsilon$. Therefore

  \begin{align*}
    U(P,f,\beta_3)-L(P,f,\beta_3)=\frac{M_2}{2}-\frac{m_2}{2}+\frac{M_3}{2}-\frac{m_2}{2} \leq \frac{2\varepsilon}{2}+\frac{2\varepsilon}{2}=2\varepsilon
  \end{align*}

  So $U(P,f,\beta_3)-L(P,f,\beta_3)$ goes to 0 and hence $f\in\mathscr R(\beta_3)$.

  \vspace{1cm}

  {\Large \color{Sepia} (d.) If $f$ is continuous at 0 prove that

  \begin{align*}
    \int f \ d\beta_1 = \int f\ d\beta_2 = \int f\ d\beta_3 = f(0)
  \end{align*}}

  \vspace{1cm}

  Since $f$ is continuous then we have that $\int f \ d\beta_3= f(0)$ as we showed in (c.).  But also continuity at 0 implies $f(0+)=f(0)$ and hence by part (a.) we have $\int f\ d\beta_2 = f(0)$.  And since $f(0-)=f(0)$ then $\int f \ d\beta_1 = f(0)$.  Therefore

  \begin{align*}
    \int f \ d\beta_1 = \int f\ d\beta_2 = \int f\ d\beta_3 = f(0)
  \end{align*}

  \pagebreak
  {\Large \color{Sepia} Problem 9. Rudin page 138 problem 5. Suppose $f$ is a bounded real function on $[a,b]$ and $f^2\in\mathscr R(\alpha)$ on $[a,b]$.  Does it follow that $f\in\mathscr R(\alpha)$?  Does the answer change if we assume $f^3\in\mathscr R(\alpha)$?}

  \vspace{1cm}

  $f$ may not be integrable even if $f^2$ is. Take ${\bf 1}_{\mathbb Q}$ to be the Dirichlet function. Then set $f(x)=2{\bf 1}_{\mathbb Q}(x)-1$, and then we have that $f$ is not integrable. If it were then we would have $\int_0^1 f \ d\alpha = 2\int_{0}^1{\bf 1}_{\mathbb Q}\ d\alpha -1$ and therefore $\int_{0}^1 {\bf 1}_{\mathbb Q} \ d\alpha = \frac{\int_{0}^1 f \ d\alpha + 1}{2}$ exists, which we know it does not. On the other hand $f^2(x)=1$, the constant function 1, which is integrable.

  For the second part we use the theorem that compositions of continuous with integrable functions is always integrable (theorem 6.11 in the text).  Set $g(x)=[f(x)]^3$ and $\phi(x)=\sqrt[3]x$.  These satisfy the conditions of the theorem since $g\in\mathscr R(\alpha)$ on $[a,b]$, and $g^3$ is bounded since $g$ is.  Moreover $\phi$ is continuous everywhere. Then $\phi\circ g = f \in\mathscr R(\alpha)$ on $[a,b]$.

  \pagebreak
  {\Large \color{Sepia} Problem 10. Rudin page 138 problem 8. Suppose $f\in\mathscr R(\alpha)$ on $[a,b]$ for every $b>a$ where $a$ is fixed.  Define

  \begin{align*}
    \int_a^\infty f(x) \ dx = \lim_{b\to \infty}\int_a^b f(x)\ dx
  \end{align*}

  if this limit exists (and is finite).  In that case, we say that the integral on the left {\it converges}.  If it also converges after $f$ has been replaced by $|f|$, it is said to {\it converge absolutely}.

  Assume that $f(x)\geq 0$ and that $f$ decreases monotonically on $[1,\infty)$.  Prove that

  \begin{align*}
    \int_1^\infty f(x) \ dx
  \end{align*}

  converges if and only if

  \begin{align*}
    \sum_{n=1}^\infty f(n)
  \end{align*}


  converges.  (This is the so-called ``integral test''.)

  }

  \vspace{1cm}

  For each $N\in\mathbb N$ we set $P_N=\{x_0=1,x_1=2,\dots,x_N=N+1\}$.  Notice that on each $[x_{n-1},x_n]$ we have that $m_n=f(x_n)=f(n+1)$ since $f$ is decreasing.  Likewise $M_n=f(x_{n-1})=f(n)$ for the same reason.  And of course $\Delta x_n=1$.  Therefore 

  \begin{align*}
    U(P_N,f)&=\sum_{n=1}^N m_n\Delta x_n=\sum_{n=1}^N f(n+1) =\sum_{n=2}^{N+1} f(n)\geq \int_1^N f \ dx\\\\
    L(P_n,f)&=\sum_{n=1}^N f(n) \leq \int_1^N f \ dx
  \end{align*}

  Now suppose $\int_1^\infty f \ dx$ converges. Since $f\geq 0$ then by the monotone convergence theorem we only need to show that the partial sums $\displaystyle \sum_{n=1}^N f(n)$ are bounded above.  But since $f\geq 0$ then  $\int_1^N f \ dx \leq \int_1^\infty f \ dx$ and therefore 

  \begin{align*}
    \sum_{n=1}^N f(n)\leq \int_1^N f\ dx \leq \int_1^\infty f\ dx 
  \end{align*}
  
  where the right-hand side is a finite real number.  Hence $\displaystyle\sum_{n=1}^\infty f(n)$ converges.  

  Now suppose $\displaystyle\sum_{n=1}^{\infty}f(n)$ converges.  Therefore $\displaystyle\sum_{n=1}^{\infty}f(n)$ converges, and so for each $N $

  \begin{align*}
    \sum_{n=2}^\infty f(n) \geq \sum_{n=2}^{N+1} f(n) \geq \int_1^N f \ dx
  \end{align*}

  for each $N$, hence $\int_1^N f \ dx$ is bounded above for each $N$.  Again because $f\geq 0$ this is a monotonically increasing sequence and so by the monotone convergence theorem, the sequence converges.  Hence $\int_1^\infty f \ dx$ converges.  

  \pagebreak

  {\Large \color{Sepia} Problem 11. Prove the Cauchy-Schwarz inequality: If $f,g\in\mathscr R(\alpha,[a,b])$ and $f,g\geq 0$ then \linebreak $\int fg\ d\alpha \leq \left(\int f^2\ d\alpha\right)^{\frac 1 2}\left(\int g^2\ d\alpha\right)^{\frac 1 2}$. }

  \vspace{1cm}

  Put $A=\int_a^b f^2 \ d\alpha, B=\int_a^b g^2\ d\alpha, C=\int_a^b fg \ d\alpha$.  Then

  \begin{align*}
    \int_a^b (Bf-Cg)^2\ d\alpha &= \int_a^b B^2f^2\ d\alpha -2 \int_a^b BCfg\ d\alpha + \int_a^b C^2g^2 \\\\
    &= B^2 \int_a^b f^2 \ d\alpha -2BC\int_a^b fg\ d\alpha + C^2\int_a^b g^2 \ d\alpha \\\\
    &=B^2A - 2BC^2 + C^2B \\\\
    &= B(BA - C^2)
  \end{align*}

  Now since $(Bf-Cg)^2\geq 0$ then we must have $B(BA^2-C^2)\geq 0$.  Now if $B=0$ then $f^2=0$ and therefore $f=0$.  So assume $B\ne 0$ and therefore $BA^2 - C^2 \geq 0$ so that

  \begin{align*}
    \left(\int_a^b g^2 \ d\alpha\right) \left(\int_a^bf^2 \ d\alpha\right)-\left(\int_a^b fg \ d\alpha\right)^2 \geq 0 \quad \Rightarrow \\\\
    \int_a^b fg\ d\alpha \leq \left(\int_a^b f^2\ d\alpha\right)^{1/2}\left(\int_a^b g^2\ d\alpha\right)^{1/2}
  \end{align*}

  \pagebreak
  {\Large \color{Sepia} Problem 12. Rudin page 139 problem 11. Let $\alpha$ be a fixed increasing function on $[a,b]$.  For $u\in\mathscr R(\alpha)$, define

  \begin{align*}
    ||u||_2 = \left\{\int_a^b |u|^2\ d\alpha\right\}^{1/2}
  \end{align*}

  Suppose $f,g,h\in\mathscr R(\alpha)$ and prove the triangle inequality

  \begin{align*}
    ||f-h||_2\leq ||f-g||_2+||g-h||_2
  \end{align*}

  as a consequence of the Schwarz inequality.

  }

  \vspace{1cm}

  We start by proving the more ``direct" triangle inequality $||f+g||_2\leq ||f||_2+||g||_2$.

  \begin{align*}
    ||f+g||_2^2 &= \int_a^b |f+g|^2 \ d\alpha \\\\
    &\leq \int_a^b (|f|+|g|)^2\ d\alpha \\\\
    &= \int_a^b |f|^2\ d\alpha + \int_a^b 2|fg|\ d\alpha + \int_a^b |g|^2\ d\alpha \\\\
    &\leq ||f||_2^2+2\left\{\int_a^b f^2 \ d\alpha\right\}^{1/2}\left\{\int_a^b g^2 \ d\alpha\right\}^{1/2}+||g||_2^2 \\\\
    &= ||f||_2^2 + 2||f||_2||g||_2 + ||g_2||^2 \\\\
    &= (||f||_2+||g||_2)^2
  \end{align*}

  Square-rooting both sides, we obtain $||f+g||_2\leq ||f||_2+||g||_2$.  With this in hand we can infer the desired inequality as follows.

  \begin{align*}
    ||f-h||_2=||f-g+g-h||_2\leq ||f-g||_2+||g-h||_2
  \end{align*}

  \pagebreak
  {\Large \color{Sepia} Problem 13. Rudin page 139 problem 10abc. Let $p,q\in \mathbb R^+$ such that

  \begin{align*}
    \frac 1 p + \frac 1 q = 1
  \end{align*}

  Prove the following.

  (a) If $u,v\geq 0$ then

  \begin{align*}
    uv\leq \frac{u^p}{p} + \frac{v^q}{q}
  \end{align*}

  Equality holds if and only if $u^p=v^q$.

  }

  \vspace{1cm}

  Consider the function $f(u)=\frac{u^p}{p}+\frac{v^q}{q}-uv$ with $q$ fixed, which has derivative $f'(u)=u^{p-1}-v$.  If $f$ has extrema then they are located where

  \begin{align*}
    0=u^{p-1}-v \quad \Rightarrow \quad u = v^{1/(p-1)}
  \end{align*}

  To show that this point is indeed a minimum, we note that since $\frac 1 p + \frac 1 q = 1$ then $1 +\frac p q = p$ and therefore $p-1=\frac p q$ and this is a positive quantity.  Hence $f'(u)=u^{p-1}-v$ is a strictly increasing function on $[0,\infty)$.  Therefore if $x\in[0,v^{1/(p-1)})$ then $f(x)>f(v^{1/(p-1)})$, and on the other hand if $x\in (v^{1/(p-1)},\infty]$ then $f(v^{1/(p-1)})<f(x)$.  Now we compute

  \begin{align*}
    f(v^{1/(p-1)}) = \frac{(v^{1/(p-1)})^p}{p}+\frac{v^q}{q}-(v^{1/(p-1)})v
  \end{align*}

  Since we have already established that $\frac p q=p-1$ then we have that $\frac{1}{p-1}=\frac q p$.  Hence the above equals

  \begin{align*}
    f(v^{1/(p-1)})=\frac{v^q}{p}+\frac{v^q}{q} - v^{1+1/(p-1)} &= v^q \left(\frac 1 p +\frac 1 q\right) - v^{p/(p-1)}\\\\
    &= v^q -v^q = 0
  \end{align*}

  From this it follows that for any $u\in[0,\infty)$ we have

  \begin{align*}
    f(v^{1/(p-1)}) = 0 \leq \frac{u^p}{p}+\frac{v^q}{q}-uv
  \end{align*}

  and therefore, as desired, we have $uv\leq \frac{u^p}{p}+\frac{v^q}{q}$.

  \vspace{1cm}

  Now if $u^p=v^q$ then $u=v^{q/p}=v^{1/(p-1)}$.  Therefore $f(u)=0=\frac{u^p}{p}+\frac{v^q}{q}$. Conversely, if $u^p\ne v^{q}$ then it follows that $u\ne v^{1/(p-1)}$.  Since $f$ has a unique minimum at $v^{1/(p-1)}$ then we must have

  \begin{align*}
    0 \ne f(u) = \frac{u^p}{p}+\frac{v^q}{q}-uv
  \end{align*}

  which entails $uv\ne \frac{u^p}{p}+\frac{v^q}{q}$.

  \vspace{1cm}

  {\Large \color{Sepia} (b.) If $f,g \in \mathscr R(\alpha), f\geq 0,g\geq 0$ and

  \begin{align*}
    \int_a^b f^p\ d\alpha = 1 = \int_a^b g^q \ d\alpha
  \end{align*}

  then

  \begin{align*}
    \int_a^b fg \ d\alpha \leq 1
  \end{align*}

  }

  \vspace{1cm}

  From part (a.) setting $u=f(x)$ and $v=g(x)$ then

  \begin{align*}
    \int_a^b fg \ d\alpha &\leq \int_a^b \left(\frac{f^p}{p}+\frac{g^q}{q}\right) \ d\alpha \\\\
    &= \frac 1 p \int_a^b f^p \ d\alpha + \frac 1 q \int_a^b g^q \ d\alpha \\\\
    &= \frac 1 p +\frac 1 q = 1
  \end{align*}

  \vspace{1cm}

  {\Large \color{Sepia} (c.) If $f,g$ are complex functions in $\mathscr R(\alpha)$ then

  \begin{align*}
    \left|\int_a^b fg\ d\alpha\right| \leq \left\{\int_a^b|f|^p\ d\alpha\right\}^{1/p}\left\{\int_a^b|g|^q\ d\alpha\right\}^{1/q}
  \end{align*}

  This is Holder's inequality, a generalization of the Schwarz inequality.
  }

  \vspace{1cm}

  Define $||f||_p=\left(\int_a^b|f|^p\ d\alpha\right)^{1/p}$ and set $\hat f = \frac{|f|}{||f||_p}, \hat g = \frac{|g|}{||g||_p}$.  Then

  \begin{align*}
    \int_a^b \hat f \hat g \ d\alpha = \frac{1}{||f||_p||g||_p}\int_a^b |fg| \ d\alpha \leq 1
  \end{align*}

  Therefore

  \begin{align*}
    \left|\int_a^b f g \ d\alpha \right| &\leq \int_a^b |fg| \ d\alpha \\\\
    &\leq ||f||_p||g||_g \\\\
    &= \left\{\int_a^b|f|^p\ d\alpha\right\}^{1/p}\left\{\int_a^b|g|^q\ d\alpha\right\}^{1/q}
  \end{align*}

  \pagebreak
  {\Large \color{Sepia} Problem 14. Rudin page 139 problem 12. With the notation of Exercise 11, suppose $f\in\mathscr R(\alpha)$ and $\varepsilon >0$.  Prove that there exists a continuous function $g$ on $[a,b]$ such that $||f-g||_2<\varepsilon$.

  {\it Hint:} Let $P=\{x_0,\dots,x_n\}$ be a suitable partition of $[a,b]$, define

  \begin{align*}
    g(t)=\frac{x_i-t}{\Delta x_i}f(x_{i-1})+\frac{t-x_{i-1}}{\Delta x_i}f(x_i)
  \end{align*}

  if $x_{i-1}\leq t \leq x_i$.

  }

  \vspace{1cm}

  Let $\varepsilon\in\mathbb R^+$ and set $P=\{x_0=a,\dots,x_n\}$ a partition such that

  \begin{align*}
    U(P,f,\alpha)-L(P,f,\alpha)<\varepsilon
  \end{align*}

  Now define $g$ as in the hint.  Clearly $g(t)$ is continuous at each point $x_{i-1}<t<x_i$ for each $i=1, \dots, n$, since it is a linear function of $t$.  So to show that $g$ is continuous on $[a,b]$ we only need to see that $g$ is continuous at each $x_i$ for $i=0,\dots,n$.  For any $i=1,\dots,n-1$ we can see that

  \begin{align*}
    \lim_{t\to x_i^-}g(t) &= \frac{x_i-x_i}{\Delta x_i}f(x_{i-1})+\frac{x_i-x_{i-1}}{\Delta x_i}f(x_i)=f(x_i) \\\\
    \lim_{t\to x_i^+}g(t) &= \frac{x_{i+1}-x_i}{\Delta x_{i+1}}f(x_{i})+\frac{x_i-x_{i}}{\Delta x_{i+1}}f(x_{i+1})=f(x_i) \\\\
    g(x_i)&=\frac{x_i-x_i}{\Delta x_i}f(x_{i-1})+\frac{x_i-x_{i-1}}{\Delta x_i}f(x_i)=f(x_i)
  \end{align*}

  so that $g$ is continuous at all of these points.  When $i=0$ we consider only $g(x_i)=\displaystyle\lim_{t\to x_i^+}g(t)$ by the same calculation as above.  Therefore $g$ is continuous at $x_0$.  When $i=n$ we consider only $g(x_i)=\displaystyle\lim_{t\to x_i^-}g(t)$ and therefore $g$ is continuous at $x_n$.

  Since $g$ is continuous then $g\in\mathscr R(\alpha,[a,b])$.  Therefore $|f-g|^2\in\mathscr R(\alpha,[a,b])$.

  Now to show that $||f-g||_2<\varepsilon$ we need to show

  \begin{align*}
    \int_a^b |f-g|^2 \ d\alpha < \varepsilon
  \end{align*}

  and it will suffice if we show that $U(Q,|f-g|^2,\alpha)<\varepsilon$ for some partition $Q$.  Define $M=\sup f,m=\inf f$ where the supremum and infimum are taken over $[a,b]$. We choose a partition $Q$ such that $U(Q,f,\alpha)-L(Q,f,\alpha)<\varepsilon/(M-m)$.  From this it follows that 

  \begin{align*}
    \sum_{i=1}^n(M_i-m_i)\Delta\alpha_i &< \frac{\varepsilon}{M-m} \quad \Rightarrow \\\\
    \sum_{i=1}^n(M-m)(M_i-m_i)\Delta\alpha_i &< \varepsilon
  \end{align*}

  We then observe that $\displaystyle\sum_{i=1}^n(M_i-m_i)^2 \Delta \alpha_i \leq \sum_{i=1}^n(M-m)(M_i-m_i)\Delta\alpha_i$. Moreover for each $i=1, \dots, n$, we can see that on the interval $[x_{i-1},x_i]$ both $g(x_{i-1})=f(x_{i-1})$ and $g(x_i)=f(x_i)$.  Further we can see that for each $t\in [x_{i-1},x_i]$ the value $g(t)$ is between $g(x_{i-1})=f(x_{i-1})$ and $g(x_i)=f(x_i)$.  So we have that $m_i\leq g(t) \leq M_i$ and therefore

  \begin{align*}
    |f(t)-g(t)|\leq M_{i}-m_i 
  \end{align*}

  and therefore 

  \begin{align*}
    |f(t)-g(t)|^2 \leq (M_i-m_i)^2
  \end{align*}

  which then implies that
  
  \begin{align*}
    U(Q,|f-g|^2, \alpha) &= \sum_{i=1}^{n}\left(\sup_{i}|f-g|^2\right)\Delta\alpha_i \\\\
    &\leq \sum_{i=1}^{n}(M_i-m_i)^2\Delta\alpha_i < \varepsilon
  \end{align*}

  Therefore 

  \begin{align*}
    ||f-g||_2=\int_a^b|f-g|^2\ d\alpha \leq U(Q,|f-g|^2, \alpha) < \varepsilon
  \end{align*}


\end{document}
