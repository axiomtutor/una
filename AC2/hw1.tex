\documentclass{article}
\usepackage{amsmath, amssymb}
\usepackage[dvipsnames]{xcolor}
\usepackage{wasysym}

\DeclareMathAlphabet{\mymathbb}{U}{BOONDOX-ds}{m}{n}

\begin{document}
  \begin{center} \Large
    MA-652 Advanced Calculus\\
    Homework 1, Jan. 9 \\
    Adam Frank
  \end{center}

  \vspace{1cm}

  {\Large \color{Sepia} Problem 1.  Let $f:[a,b]\to \mathbb R$ be differentiable at $x\in (a,b)$ and $k\in \mathbb R$.  Prove that $(kf)'(x) = kf'(x)$.}

  \vspace{1cm}

  If we define the difference quotient at $x\in (a,b)$,

  \begin{align*}
    \phi(x) = \frac{(kf)(t)-(kf)(x)}{t-x}
  \end{align*}

  then our task is to find $\displaystyle\lim_{t\to x}\phi(x)$.  But since

  \begin{align*}
    \lim_{t\to x}\phi = \lim_{t\to x}\frac{kf(t)-kf(x)}{t-x}
  \end{align*}

  by definition of multiplying functions, then this is

  \begin{align*}
    \lim_{t\to x}k\frac{f(t)-f(x)}{t-x} = k\lim_{t\to x}\frac{f(t)-f(x)}{t-x} = kf'(x)
  \end{align*}

  where this limit is guaranteed to exist by the differentiability of $f$ in this interval.

  \pagebreak

  %%%

  {\Large \color{Sepia} Problem 2.  Let $f,g:[a,b]\to \mathbb R$ be differentiable at $x\in (a,b)$.

  (a). Prove the quotient rule using the limit definition, wherever the denominator isn't 0.}

  \vspace{1cm}

  For any $x\in (a,b)$ such that $g(x)\ne 0$ we will show that $\left(\frac{f}{g}\right)'(x) = \frac{f'(x)g(x)-f(x)g'(x)}{[g(x)]^2}$.  We define the difference quotient

  \begin{align*}
    \phi(t) = \frac{f(t)/g(t)-f(x)/g(x)}{t-x} = \frac{\frac{f(t)g(x)-f(x)g(t)}{g(t)g(x)}}{t-x}
  \end{align*}

  Therefore

  \begin{align*}
    \left(\frac f g\right)'(t) &= \lim_{t\to x}\frac{f(t)g(x)-f(x)g(t)}{[g(t)g(x)](t-x)} \\\\
    &= \frac{1}{g(x)}\lim_{t\to x}\frac{1}{g(t)}\left(\frac{f(t)g(x)-f(x)g(x)+f(x)g(x)-f(x)g(t)}{t-x}\right) \\\\
    &= \frac{1}{g(x)}\lim_{t\to x}\frac{1}{g(t)}\cdot \lim_{t\to x}\left(\frac{f(t)-f(x)}{t-x}\cdot g(x) - f(x)\frac{g(t)-g(x)}{t-x}\right) \\\\
    &= \frac{1}{[g(x)]^2}(f'(x)g(x)-f(x)g'(x))
  \end{align*}

  The final equation follows because we assumed that $g(x)\ne 0$ and therefore $\displaystyle\lim_{t\to x}\frac{1}{g(t)}=\frac{1}{g(x)}$.  Also we assumed both functions are differentiable in the interval and hence $\displaystyle\lim_{t\to x}\frac{f(t)-f(x)}{t-x}=f'(x)$ and also $\displaystyle\lim_{t\to x}\frac{g(t)-g(x)}{t-x}=g'(x)$.  The proof is then complete.

  \vspace{1cm}

  {\Large (b).  Use the limit definition to find the derivative of $\frac{1}{x}$.}

  \vspace{1cm}

  \begin{align*}
    \lim_{t\to x} \frac{\frac{1}{t}-\frac 1 x}{t-x} &= \lim_{t\to x}\frac{\frac{x-t}{tx}}{t-x}\\\\
    &= -\lim_{t\to x}\frac{1}{tx} \\\\
    &= -\frac{1}{x^2}
  \end{align*}

  \vspace{1cm}

  {\Large (c).  Use (b) with the chain rule to derive the quotient rule.}

  \vspace{1cm}

  \begin{align*}
    \left(\frac f g \right)'(x) &= \left(f(x)\cdot \frac{1}{g(x)}\right)' \\\\
    &= f'(x)\cdot \frac{1}{g(x)} + f(x)\left[\frac{1}{g(x)}\right]'
  \end{align*}

  by the product rule.  Now by the chain rule

  \begin{align*}
    \left[\frac 1 {g(x)} \right]' = -\frac{1}{[g(x)]^2}g'(x)
  \end{align*}

  So we can infer from these two equations that

  \begin{align*}
    \left(\frac f g \right)'(x) &= \frac{f'(x)}{g(x)}-f(x)\frac{g'(x)}{[g(x)]^2} \\\\
    &= \frac{f'(x)g(x)-f(x)g'(x)}{[g(x)]^2}
  \end{align*}

  \pagebreak

  {\Large \color{Sepia} Problem 3. Rudin page 114 problem 1. If $f$ is defined on $\mathbb R$ and $\forall x,y \in \mathbb R$ we have $|f(x)-f(y)|\leq (x-y)^2$, then prove that $f$ is constant.}

  \vspace{1cm}

  This feels like a complex analysis theorem---this sort of thing isn't supposed to be true for real functions! \smiley{}

  To show that $f$ is constant we'll prove that the derivative is zero everywhere.  That is to say, we'll show that at every $x\in \mathbb R$

  \begin{align*}
    \left|\frac{f(t)-f(x)}{t-x}\right| < \varepsilon
  \end{align*}

  for each $\varepsilon\in\mathbb R^+$, whenever $|t-x|<\delta$ for some corresponding $\delta$.  Choose $\delta = \varepsilon$ in fact.  Then if $|t-x| < \delta$ we have

  \begin{align*}
    \left|\frac{f(t)-f(x)}{t-x}\right|\leq \frac{(t-x)^2}{|t-x|} =|t-x| < \delta = \varepsilon
  \end{align*}

  \pagebreak

  {\Large \color{Sepia} Problem 4. Rudin page 114 problem 2. Suppose $f'(x)>0$ in $(a,b)$.  Prove that $f$ is strictly increasing in $(a,b)$.  Let $g$ be its inverse.  Prove that $g$ is differentiable and

  \begin{align*}
    g'(f(x)) = \frac{1}{f'(x)}
  \end{align*}
  }

  \vspace{1cm}

  We already know that $f$ is monotonically increasing due to theorem 5.11.  Moreover, since $f'(x) >0$ we have that there can be no two values $c,d\in (a,b)$ such that $c < d$ and $f(c)=f(d)$.  If there were, by the mean-value theorem there would have to be a point $x_0\in (c,d)$ such that $f'(x) = \frac{f(d)-f(c)}{d-c} = 0$.

  Since $f$ is strictly increasing it is one-to-one and has an inverse.  By definition if $g$ is the inverse then $f(g(y)) = y$ for all $y$ in the co-domain of $f$.  By the chain rule we have

  \begin{align*}
    f'(g(y))g'(y) = 1 \quad \Rightarrow\\\\
    g'(y) = \frac{1}{f'(g(y))}
  \end{align*}

  If we regard $y = f(x)$ and hence $g(y) = x$ then we have

  \begin{align*}
    g'(f(x)) = \frac{1}{f'(x)}
  \end{align*}

  \pagebreak

  {\Large \color{Sepia} Problem 5. Rudin page 114 problem 4. If $C_0+\frac{C_1}{2}+\dots+\frac{C_n}{n+1}=0$ then

  \begin{align*}
    C_0+C_1x+\dots+C_nx^n = 0
  \end{align*}
  has at least one solution in [$0,1$].
  }

  \vspace{1cm}

  Define $F(x) = C_0x + C_1x^2/2 + \dots + C_nx^n/(n+1)$ and observe that $F'(x) = C_0+\dots+C_nx^n$.  Now $F(0)=0$ and $F(1)=C_0+\dots+\frac{C_n}{n+1}=0$.  By the mean value theorem there is a point $c\in (0,1)$ such that $F'(c) = \frac{F(1)-F(0)}{1-0} = 0$.

  \pagebreak

  {\Large \color{Sepia} Problem 6. Rudin page 114 problem 5. Suppose $f$ is defined and differentiable for every $x>0$, and $\displaystyle\lim_{x\to \infty}f'(x) = 0$.  Let $g(x)=f(x+1)-f(x)$.  Prove that $\displaystyle \lim_{x\to \infty}g(x)=0$.}

  \vspace{1cm}

  Set $\varepsilon\in\mathbb R^+$.  Let $M\in\mathbb R$ be such that $|f'(a)|<\varepsilon$ for all $a>M$.  We will show that, for this value of $M$, it follows that $|g(x)|<\varepsilon$ for all $x>M$.

  Now for any such $x>M$ we have that $x+1>M$.  So on the interval $(x,x+1)$ we have that $\frac{f(x+1)-f(x)}{x+1-x} = f(x+1)-f(x) = g(x) = f'(c)$ for some $c\in (x,x+1)$.  Therefore $|g(x)|=|f(x+1)-f(x)| = |f'(c)|<\varepsilon$.  Since $x$ was chosen arbitrarily in $(M,\infty)$, we have shown

  \begin{align*}
    \lim_{x\to \infty}g(x)=0
  \end{align*}

  \pagebreak

  {\Large \color{Sepia} Problem 7. Rudin page 114 problem 6. Suppose (a) $f$ is continuous for $x\geq 0$, (b) $f'(x)$ exists for $x>0$, (c) $f(0)=0$, (d) $f'$ is monotonically increasing.  Put $g(x)=\frac{f(x)}{x}$ for $x>0$ and prove that $g$ is monotonically increasing.}

  \vspace{1cm}

  We show that the derivative is non-negative.  Since

  \begin{align*}
    g'(x) = \frac{f'(x)x-f(x)}{x^2}
  \end{align*}

  then it suffices to show that $f'(x)x-f(x)\geq 0$ for every $x\in\mathbb R^+$.  To leverage condition (d) we apply the mean value theorem.  There must exist a $0 < c < x$ such that

  \begin{align*}
    \frac{f(x)-f(0)}{x-0} = \frac{f(x)}{x} = f'(c)
  \end{align*}

  and hence $f(x) = f'(c)x$.  Since $x > 0$, and since $f'$ is increasing and $c<x$, we must have

  \begin{align*}
    f(x) = f'(c)x \leq f'(x)x
  \end{align*}

  which implies

  \begin{align*}
    0\leq f'(x)x-f(x)
  \end{align*}

  as desired.

  \pagebreak

  {\Large \color{Sepia} Problem 8. Rudin page 117 problem 22 abc. Suppose $f$ is a real function on $\mathbb R$.  (a) If $f$ is differentiable and $f'(t)\ne 1$ for all real $t$, prove that $f$ has at most one fixed point.}

  \vspace{1cm}

  Suppose $f(a)=a$ and $f(b)=b$ with $a<b$.  Then by the mean value theorem there is a point $a<c<b$ such that $f'(c) = \frac{f(b)-f(a)}{b-a}=\frac{b-a}{b-a}=1$.

  \vspace{1cm}

  {\Large \color{Sepia} (b) Show that the function $f$ defined by

  \begin{align*}
    f(t) = t+(1+e^t)^{-1}
  \end{align*}

  has no fixed point, although $0<f'(t)<1$ for all real $t$.

  }

  \vspace{1cm}

  Suppose $f(a) = a = a+(1+e^a)^{-1}$ so that $(1+e^a)^{-1}=0$, which is impossible.  To see that $0<f'(t)<1$ note

  \begin{align*}
    f'(t)=1-(1+e^t)^{-2}e^t
  \end{align*}

  so that we need to show $0 < \frac{e^t}{(1+e^t)^2} < 1$.  The first inequality is clear.  The second is equivalent to

  \begin{align*}
    e^t &< 1+2e^t +e^{2t} \quad \Leftrightarrow \\\\
    0 &< 1+e^t+e^{2t}
  \end{align*}

  where this last is clearly a sum of positive quantities for each real $t$.

  \vspace{1cm}

  {\Large \color{Sepia} (c) However, if there is a constant $A<1$ such that $|f'(t)|\leq A$ for all real $t$, prove that a fixed point $x$ of $f$ exists, and that $x=\lim x_n$ where $x_1$ is arbitrary, and $x_{n+1}=f(x_n)$ for $n=1,2,\dots$}

  \vspace{1cm}

  First we show that the sequence is Cauchy so that the limit exists.  Let $\varepsilon\in \mathbb R^+$.  We first establish that the sequence values get closer by showing that for any $n\geq 1$ we have

  \begin{align*}
    |x_{n+1}-x_n| \leq A^{n-1}|x_2-x_1|
  \end{align*}

  The proof is by induction on $n$.  The base-case is trivial since it's $|x_2-x_1|\leq |x_2-x_1|$.  Now if the claim holds for $n\geq 1$ then apply the mean value theorem to $x_{n}$ and $x_{n+1}$, so that there exists a $c$ between them such that

  \begin{align*}
    \frac{f(x_{n+1})-f(x_n)}{x_{n+1}-x_{n}} = f'(c)
  \end{align*}

  Therefore

  \begin{align*}
    |f(x_{n+1})-f(x_n)| &= |x_{n+2}-x_{n+1}| = |f'(c)||x_{n+1}-x_n| \\\\
    &\leq A(A^{n-1})|x_2-x_1| = A^n|x_2-x_1|
  \end{align*}

  Now we can consider that, if $m,n\in\mathbb N$ with $m \leq n$ then

  \begin{align*}
    |x_n-x_m| &\leq |x_{m+1}-x_m|+\dots+|x_n-x_{n-1}| \\\\
    &= (A^{m-1}+A^{m}+\dots+A^{n-2})|x_2-x_1| \\\\
    &= \left(\frac{1-A^{n-1}}{1-A}-\frac{1-A^{m-1}}{1-A}\right)|x_2-x_1| \\\\
    &= (A^{m-1}-A^{n-1})\frac{|x_2-x_1|}{1-A}
  \end{align*}

  So finally we can say, corresponding to $\varepsilon$, we choose $N\in\mathbb N$ such that $A^N < \left(\frac{1-A}{|x_2-x_1|}\right)\varepsilon$.  Then it follows from the above that

  \begin{align*}
    |x_n-x_m| &\leq (A^{m-1}-A^{n-1})\frac{|x_2-x_1|}{1-A} \\\\
    &\leq A^N\frac{|x_2-x_1|}{1-A} \\\\
    &< \left(\frac{1-A}{|x_2-x_1|}\right)\varepsilon \cdot \left(\frac{|x_2-x_1|}{1-A}\right) = \varepsilon
  \end{align*}

  Since we've now concluded that the sequence is Cauchy, we go on to compute its limit.

  \begin{align*}
    x = \lim_{n\to \infty}x_n = \lim_{n\to \infty}f(x_{n-1}) = f\left(\lim_{n\to \infty}x_{n-1}\right) = f(x)
  \end{align*}

  The third equation is due to the assumption that $f$ is differentiable and therefore continuous.  This demonstrates that $x$ is a fixed point of $f$.

  \pagebreak

  {\Large \color{Sepia} Problem 9. Rudin page 119 problem 26. Suppose $f$ is differentiable on $[a,b]$, $f(a)=0$, and there is a real number $A$ such that $|f'(x)|\leq A|f(x)|$ on $[a,b]$.  Prove that $f(x)=0$ for all $x\in[a,b]$.  {\it Hint:} Fix $x_0\in[a,b]$, let

  \begin{align*}
    M_0 = \sup|f(x)|, \quad M_1 = \sup |f'(x)|
  \end{align*}

  for $a\leq x\leq x_0$.  For any such $x$,

  \begin{align*}
    |f(x)|\leq M_1(x_0-a)\leq A(x_0-a)M_0
  \end{align*}

  Hence $M_0=0$ if $A(x_0-a)<1$.  That is, $f=0$ on $[a,x_0]$.

  }

  \vspace{1cm}

  From the mean value theorem, for any $x\in [a,x_0]$, we have some $c\in (a,x)$ such that

  \begin{align*}
    \frac{f(x)-f(a)}{x-a} = f'(x)
  \end{align*}

  and so

  \begin{align*}
    \left|\frac{f(x)}{x-a}\right| = |f'(c)| \leq M_1
  \end{align*}

  and so

  \begin{align*}
    |f(x)|\leq M_1(x-a)\leq M_1(x_0-a)
  \end{align*}

  since $M_1\geq 0$ and $x_0>x$.  All that remains to show for the hint is that $M_1\leq AM_0$ but this follows from the properties of suprema, in particular that $\sup{cX}=c\sup X$ where $X$ is a set of real numbers and $c$ is a non-negative constant.  In this particular application we take $X$ to be the range of $f$ on $[a,x_0]$.

  Now that we have proved the chain of inequalities in the hint, we have that $|f(x)|\leq A(x_0-a)M_0$.  We argue that $A(x_0-a)<1$ for some choice of $x_0$.  But since $A$ is constant then a choice of $x_0$ sufficiently close to $a$ clearly exists.  Therefore on this interval $|f(x)|\leq A(x_0-a)M_0\leq M_0$.  But also we have that $A(x_0-a)M_0$ is an upper bound on $|f(x)|$ for all $a < x < x_0$ so that $M_0\leq A(x_0-a)M_0$ and hence $A(x_0-a)M_0=M_0$.  Given that we have already established $A(x_0-a)<1$ this can only be true if $M_0=0$, and this immediately implies that $f=0$ on $[a,x_0]$.

  Now given that this holds, we repeat the argument above, this time on the interval $[x_0,b]$.  We will again find that if $2x_0 \leq b$ then on $[x_0,2x_0]$ we have that $f=0$.  Proceeding likewise, we continue until $nx_0>b$ for some $n\in\mathbb N$.  But in this case, we repeat the proof but with $x_0$ replaced by $b-nx_0$.  Here again we find that $f=0$ on $[(n-1)x_0,b]$.

  The above then shows that on $[a,x_0]\cup[x_0,x_1]\cup\dots\cup[x_{n-1},b]=[a,b]$ we have $f=0$.

  \pagebreak

  {\Large \color{Sepia} Problem 10. Rudin page 119 problem 27. Let $\phi$ be a real function defined on a rectangle $R$ in the plane, given by $a\leq x\leq b$, $\alpha\leq y\leq \beta$.  A {\it solution} of the initial-value problem

  \begin{align*}
    y' = \phi(x,y), \quad y(a)=c, \quad (\alpha\leq c\leq \beta)
  \end{align*}

  is, by definition, a differentiable function $f$ on $[a,b]$ such that $f(a)=c$, $\alpha\leq f(x)\leq \beta$ and

  \begin{align*}
    f'(x)=\phi(x,f(x)) \quad (a\leq x\leq b)
  \end{align*}

  Prove that such a problem has at most one solution if there is a constant $A$ such that

  \begin{align*}
    |\phi(x,y_2)-\phi(x,y_1)|\leq A|y_2-y_1|
  \end{align*}

  whenever $(x,y_1)\in R$ and $(x,y_2)\in R$.

  {\it Hint:} Apply exercise 26 to the difference of two solutions.  Note that this uniqueness theorem does not hold for the initial-value problem

  \begin{align*}
    y'=y^{1/2}, \quad y(0)=0
  \end{align*}

  which has two solutions: $f(x)=0$ and $f(x)=x^2/4$.  Find all other solutions.

  }

  \vspace{1cm}

  Suppose $f_1, f_2$ are two solutions, and define $h(x)=f_1(x)-f_2(x)$.  So if I can show that $h(x)$ is the zero function, this implies that $f_1=f_2$. We have that  $h'(x) = (f_1(x)-f_2(x))' = \phi(x,f_1(x))-\phi(x,f_2(x))$.  Then

  \begin{align*}
    |h'(x)| = |(f_1-f_2)'(x)| = |\phi(x,f_1(x))-\phi(x,f_2(x))| \leq A|f_1(x)-f_2(x)| = A|h(x)|
  \end{align*}

  and so by the previous exercise, $h=f_1-f_2=0$ and so $f_1=f_2$.  This shows that there is at most one solution.

  \vspace{1cm}

  To find all solutions to the differential equaiton

  \begin{align*}
    y'=y^{1/2}\qquad y(0)=0
  \end{align*}

  we first set $f$ to be some solution to this equation, and assume that it is not zero.  Then $f' = f^{1/2}$ and also $f'' = \frac{1}{2}f^{-1/2}f'=\frac 1 2 (f^{-1/2}f^{1/2})=1/2$.  This then shows that $f' = \frac x 2 + C$ and therefore

  \begin{align*}
    f = \frac{x^2}{4}+Cx+D
  \end{align*}

  And since $f(0)=0=\frac{0^2}{4}+C(0)+D = D$ then we can further state that every function of the form

  \begin{align*}
    f(x) = \frac{x^2}{4}+Cx
  \end{align*}

  is a solution.

  \pagebreak



































\end{document}
