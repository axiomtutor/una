\documentclass{article}
\usepackage{amsmath, amssymb}
\usepackage[dvipsnames]{xcolor}
\usepackage{wasysym}

\DeclareMathAlphabet{\mymathbb}{U}{BOONDOX-ds}{m}{n}

\begin{document}
  \begin{center} \Large
    MA-652 Advanced Calculus\\
    Homework 1, Jan. 9 \\
    Adam Frank
  \end{center}

  \vspace{1cm}

  {\Large \color{Sepia} Problem 1.  Let $f:[a,b]\to \mathbb R$ be differentiable at $x\in (a,b)$ and $k\in \mathbb R$.  Prove that $(kf)'(x) = kf'(x)$.}

  \vspace{1cm}

  If we define the difference quotient at $x\in (a,b)$,

  \begin{align*}
    \phi(x) = \frac{(kf)(t)-(kf)(x)}{t-x}
  \end{align*}

  then our task is to find $\displaystyle\lim_{t\to x}\phi(x)$.  But since

  \begin{align*}
    \lim_{t\to x}\phi = \lim_{t\to x}\frac{kf(t)-kf(x)}{t-x}
  \end{align*}

  by definition of multiplying functions, then this is

  \begin{align*}
    \lim_{t\to x}k\frac{f(t)-f(x)}{t-x} = k\lim_{t\to x}\frac{f(t)-f(x)}{t-x} = kf'(x)
  \end{align*}

  where this limit is guaranteed to exist by the differentiability of $f$ in this interval.

  \pagebreak

  %%%

  {\Large \color{Sepia} Problem 2.  Let $f,g:[a,b]\to \mathbb R$ be differentiable at $x\in (a,b)$.

  (a). Prove the quotient rule using the limit definition, wherever the denominator isn't 0.}

  \vspace{1cm}

  For any $x\in (a,b)$ such that $g(x)\ne 0$ we will show that $\left(\frac{f}{g}\right)'(x) = \frac{f'(x)g(x)-f(x)g'(x)}{[g(x)]^2}$.  We define the difference quotient

  \begin{align*}
    \phi(t) = \frac{f(t)/g(t)-f(x)/g(x)}{t-x} = \frac{\frac{f(t)g(x)-f(x)g(t)}{g(t)g(x)}}{t-x}
  \end{align*}

  Therefore

  \begin{align*}
    \left(\frac f g\right)'(t) &= \lim_{t\to x}\frac{f(t)g(x)-f(x)g(t)}{[g(t)g(x)](t-x)} \\\\
    &= \frac{1}{g(x)}\lim_{t\to x}\frac{1}{g(t)}\left(\frac{f(t)g(x)-f(x)g(x)+f(x)g(x)-f(x)g(t)}{t-x}\right) \\\\
    &= \frac{1}{g(x)}\lim_{t\to x}\frac{1}{g(t)}\cdot \lim_{t\to x}\left(\frac{f(t)-f(x)}{t-x}\cdot g(x) - f(x)\frac{g(t)-g(x)}{t-x}\right) \\\\
    &= \frac{1}{[g(x)]^2}(f'(x)g(x)-f(x)g'(x))
  \end{align*}

  The final equation follows because we assumed that $g(x)\ne 0$ and therefore $\displaystyle\lim_{t\to x}\frac{1}{g(t)}=\frac{1}{g(x)}$.  Also we assumed both functions are differentiable in the interval and hence $\displaystyle\lim_{t\to x}\frac{f(t)-f(x)}{t-x}=f'(x)$ and also $\displaystyle\lim_{t\to x}\frac{g(t)-g(x)}{t-x}=g'(x)$.  The proof is then complete.

  \vspace{1cm}

  {\Large (b).  Use the limit definition to find the derivative of $\frac{1}{x}$.}

  \vspace{1cm}

  \begin{align*}
    \lim_{t\to x} \frac{\frac{1}{t}-\frac 1 x}{t-x} &= \lim_{t\to x}\frac{\frac{x-t}{tx}}{t-x}\\\\
    &= -\lim_{t\to x}\frac{1}{tx} \\\\
    &= -\frac{1}{x^2}
  \end{align*}

  \vspace{1cm}

  {\Large (c).  Use (b) with the chain rule to derive the quotient rule.}

  \vspace{1cm}

  \begin{align*}
    \left(\frac f g \right)'(x) &= \left(f(x)\cdot \frac{1}{g(x)}\right)' \\\\
    &= f'(x)\cdot \frac{1}{g(x)} + f(x)\left[\frac{1}{g(x)}\right]'
  \end{align*}

  by the product rule.  Now by the chain rule

  \begin{align*}
    \left[\frac 1 {g(x)} \right]' = -\frac{1}{[g(x)]^2}g'(x)
  \end{align*}

  So we can infer from these two equations that

  \begin{align*}
    \left(\frac f g \right)'(x) &= \frac{f'(x)}{g(x)}-f(x)\frac{g'(x)}{[g(x)]^2} \\\\
    &= \frac{f'(x)g(x)-f(x)g'(x)}{[g(x)]^2}
  \end{align*}

  \pagebreak

  {\Large \color{Sepia} Problem 3. Rudin page 114 problem 1. If $f$ is defined on $\mathbb R$ and $\forall x,y \in \mathbb R$ we have $|f(x)-f(x)|\leq (x-y)^2$, then prove that $f$ is constant.}

  \vspace{1cm}

  This feels like a complex analysis theorem---this sort of thing isn't supposed to be true for real functions! \smiley{}

  To show that $f$ is constant we'll prove that the derivative is zero everywhere.  That is to say, we'll show that at every $x\in \mathbb R$

  \begin{align*}
    \left|\frac{f(t)-f(x)}{t-x}\right| < \varepsilon
  \end{align*}

  for each $\varepsilon\in\mathbb R^+$, whenever $|t-x|<\delta$ for some corresponding $\delta$.  Choose $\delta = \varepsilon$ in fact.  Then if $|t-x| < \delta$ we have

  \begin{align*}
    \left|\frac{f(t)-f(x)}{t-x}\right|<\frac{(t-x)^2}{\delta} < \frac{\delta^2}{\delta} = \varepsilon
  \end{align*}

  \pagebreak

  {\Large \color{Sepia} Problem 4. Rudin page 114 problem 2. Suppose $f'(x)>0$ in $(a,b)$.  Prove that $f$ is strictly increasing in $(a,b)$.  Let $g$ be its inverse.  Prove that $g$ is differentiable and

  \begin{align*}
    g'(f(x)) = \frac{1}{f'(x)}
  \end{align*}
  }

  \vspace{1cm}

  We already know that $f$ is monotonically increasing due to theorem 5.11.  Moreover, since $f'(x) >0$ we have that there can be no two values $c,d\in (a,b)$ such that $c < d$ and $f(c)=f(d)$.  If there were, by the mean-value theorem there would have to be a point $x_0\in (c,d)$ such that $f'(x) = \frac{f(d)-f(c)}{d-c} = 0$.

  Since $f$ is strictly increasing it is one-to-one and has an inverse.  By definition if $g$ is the inverse then $f(g(y)) = y$ for all $y$ in the co-domain of $f$.  By the chain rule we have

  \begin{align*}
    f'(g(y))g'(y) = 1 \quad \Rightarrow\\\\
    g'(y) = \frac{1}{f'(g(y))}
  \end{align*}

  If we regard $y = f(x)$ and hence $g(y) = x$ then we have

  \begin{align*}
    g'(f(x)) = \frac{1}{f'(x)}
  \end{align*}

  \pagebreak

  {\Large \color{Sepia} Problem 5. Rudin page 114 problem 4. If $C_0+\frac{C_1}{2}+\dots+\frac{C_n}{n+1}=0$ then

  \begin{align*}
    C_0+C_1x+\dots+C_nx^n = 0
  \end{align*}
  has at least one solution in [$0,1$].
  }

  \vspace{1cm}

  Define $F(x) = C_0x + C_1x^2/2 + \dots + C_nx^n/(n+1)$ and observe that $F'(x) = C_0+\dots+C_nx^n$.  Now $F(0)=0$ and $F(1)=C_0+\dots+\frac{C_n}{n+1}=0$.  By the mean value theorem there is a point $c\in (0,1)$ such that $F'(c) = \frac{F(1)-F(0)}{1-0} = 0$.

  \pagebreak

  {\Large \color{Sepia} Problem 6. Rudin page 114 problem 5. Suppose $f$ is defined and differentiable for every $x>0$, and $\displaystyle\lim_{x\to \infty}f'(x) = 0$.  Let $g(x)=f(x+1)-f(x)$.  Prove that $\lim_{x\to \infty}g(x)=0$.}

  \vspace{1cm}

  Let $M\in\mathbb R$ be such that $|f'(a)|<\varepsilon$ for all $a>M$.  We will show that, for this value of $M$, it follows that $|g(x)|<\varepsilon$ for all $x>M$.

  Now for any such $x>M$ we have that $x+1>M$.  So on the interval $(x,x+1)$ we have that $\frac{f(x+1)-f(x)}{x+1-x} = f(x+1)-f(x) = f'(c)$ for some $c\in (x,x+1)$.  Therefore $|f(x+1)-f(x)| = |f'(c)|<\varepsilon$.  Since $x$ was chosen arbitrarily in $(M,\infty)$, we have shown

  \begin{align*}
    \lim_{x\to \infty}g(x)=0
  \end{align*}

  \pagebreak

  {\Large \color{Sepia} Problem 7. Rudin page 114 problem 6. Suppose (a) $f$ is continuous for $x\geq 0$, (b) $f'(x)$ exists for $x>0$, (c) $f(0)=0$, (d) $f'$ is monotonically increasing.  Put $g(x)=\frac{f(x)}{x}$ for $x>0$ and prove that $g$ is monotonically increasing.}

  \vspace{1cm}

  We show that the derivative is non-negative.  Since

  \begin{align*}
    g'(x) = \frac{f'(x)x-f(x)}{x^2}
  \end{align*}

  then it suffices to show that $f'(x)x-f(x)\geq 0$ for every $x\in\mathbb R^+$.  To leverage condition (d) we apply the mean value theorem.  There must exist a $0 < c < x$ such that

  \begin{align*}
    \frac{f(x)-f(0)}{x-0} = \frac{f(x)}{x} = f'(c)
  \end{align*}

  and hence $f(x) = f'(c)x$.  Since $x > 0$, and since $f'$ is increasing and $c<x$, we must have

  \begin{align*}
    f(x) = f'(c)x \leq f'(x)x
  \end{align*}

  which implies

  \begin{align*}
    0\leq f'(x)x-f(x)
  \end{align*}

  as desired.

  \pagebreak

  {\Large \color{Sepia} Problem 8. Rudin page 117 problem 22 abc. }

  \vspace{1cm}

  First we need to show that based on these assumptions, $(kf)'(x)$ exists.

  \pagebreak

  {\Large \color{Sepia} Problem 9. Rudin page 119 problem 26. }

  \vspace{1cm}

  First we need to show that based on these assumptions, $(kf)'(x)$ exists.

  \pagebreak

  {\Large \color{Sepia} Problem 10. Rudin page 119 problem 27. }

  \vspace{1cm}

  First we need to show that based on these assumptions, $(kf)'(x)$ exists.

  \pagebreak



































\end{document}
