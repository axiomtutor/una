\documentclass{exam}
\usepackage{amsthm,amsmath,amssymb,fullpage}
\usepackage{marvosym}
\usepackage{graphicx}

\DeclareMathAlphabet{\mymathbb}{U}{BOONDOX-ds}{m}{n}

\begin{document}
\noindent \textbf{Name: Adam Frank}\hfill \textbf{\today}

\vspace{.3cm}
\hrule
\begin{center}
{\bf \Large{Math 637: Homework Chapter 3}}
\end{center}
\hrule
\vspace{.3cm}


\begin{questions}
\question(4.1.1)\\
Let $G$ act on the set $A$.  Prove that if $a,b\in A$ and $b=g\cdot a$ for some $g\in G$, then $G_b=gG_ag^{-1}$.  Deduce that if $G$ acts transitively on $A$ then the kernel of the action is $\displaystyle\bigcap_{g\in G} gG_ag^{-1}$.

\begin{proof}
  Let $x\in G_b$ so that $x$ fixes $b$.  Then $g^{-1}xg\cdot a = g^{-1}x\cdot b = g^{-1}\cdot b=a$.  Hence $g^{-1}xg\in G_a$ which entails $x\in gG_ag^{-1}$.

  Now let $y\in G_a$ so that $gyg^{-1}\in gG_ag^{-1}$.  Then $gyg^{-1}\cdot b = gy\cdot a = g\cdot a = b$ and so we have $gyg^{-1}\in G_b$.  So we have shown that $G_b=gG_ag^{-1}$.

  Finally consider if the action is transitive on $A$ and consider some $k\in G$ in the kernel of the action.  Let $g\in G$ and let $b=g\cdot a$.  Then $gG_ag^{-1} = G_b$ where $b = g\cdot a$.  Since $k\in G_b$ for every $b$ then we have

  \begin{align*}
    k\in \bigcap_{g\in G}gG_ag^{-1}
  \end{align*}

  So interestingly, the requirement that the action is transitive is not needed to show that the kernel is contained in $\displaystyle \bigcap_{g\in G}gG_ag^{-1}$.  Now for the reverse containment, suppose $k\in \bigcap_{g\in G}gG_ag^{-1}$, and let $b\in A$.  Since the action is transitive we let $g^*\in G$ such that $g^*\cdot a = b$.  Then $k\in g^*G_a(g^*)^{-1} = G_b$ and hence $k$ fixes $b$.  So $\displaystyle \bigcap_{g\in G}gG_ag^{-1}$ is contained in the kernel, and hence must be the kernel.
\end{proof}

\question(4.1.3)\\
Assume $G$ is an abelian, transitive subgroup of $S_A$.  Show that $\sigma(a)\ne a$ for all $\sigma\in G-\{1\}$ and $a\in A$.  Deduce that $|G|=|A|$.

\begin{proof}
  For any $a\in A$ and $\sigma\in G-\{\mymathbb 1\}$, we have $\sigma\not\in \displaystyle \bigcap_{\tau\in G}\tau G_a\tau^{-1}$ and hence there must be some particular $\tau$ such that $\sigma\not\in \tau G_a\tau^{-1}=G_a$.  This last equation follows from the fact that $G$ is abelian.  But since this now tells us that $\sigma$ does not fix $a$ then we have the desired result.

  To see that $|G|=|A|$ we take an arbitrary $a\in A$ and prove that the function $f_a(g) = g\cdot a$ is a bijection.  Because the action is transitive we already have that the function is surjective.  Now if $f(g_1)=f(g_2) = g_1\cdot a = g_2 \cdot a$ then we have $a=g_1^{-1}g_2\cdot a$ which from the above entails $g_1^{-1}g_2=\mymathbb 1$ and then $g_1=g_2$.
\end{proof}

\question(4.1.10(a))\\
Let $H$ and $K$ be subgroups of $G$.  For each $x\in G$ define the $HK$ double coset of $x$ in $G$ to be

\begin{align*}
  HxK = \{hxk|h\in H, k\in K\}
\end{align*}

Prove that $HxK$ is the union of the left cosets $x_1K, \dots, x_nK$ where $\{x_1K, \dots, x_nK\}$ is the orbit containing $xK$ of $H$ acting by left multiplication on the set of left cosets of $K$.

\begin{proof}
  To show that $HxK$ is the union of $x_1K, \dots, x_nK$, let $hxk\in HxK$.  We need to find some $x_iK$ such that $hxk\in x_iK$.  However, the $x_1K, \dots, x_nK$ is the orbit of $H$ acting on $xK$ by left multiplication, so $h\cdot xK = (hx)K = x_iK$ for some $i$, and so there are $k'$ and $k''$ such that $hxk' = x_ik''$.  Then

  \begin{align*}
    hx & = x_ik''(k')^{-1} \implies \\
    hxk &= x_ik''(k')^{-1}k \in x_iK
  \end{align*}

  and we are done.
\end{proof}

\question(4.2.2)\\
List the elements of $S_3$ as $1, (1\ 2), (2\ 3), (1\ 3), (1\ 2\ 3), (1\ 3\ 2)$ and label these with the integers 1, 2, 3, 4, 5, 6, respectively.  Exhibit the image of each element of $S_3$ under the left regular representation of $S_3$ into $S_6$.

{\it Calculation:} To find the image of 1 we find the action of 1 on all of the group elements:

\begin{align*}
  1\mapsto
  \begin{pmatrix}
  1 & (1\ 2) & (2\ 3) & (1\ 3) & (1\ 2\ 3) & (1\ 3\ 2) \\
  1 & (1\ 2) & (2\ 3) & (1\ 3) & (1\ 2\ 3) & (1\ 3\ 2)
  \end{pmatrix}
\end{align*}

or written with the given labels,

\begin{align*}
  1\mapsto
  \begin{pmatrix}
  1 & 2 & 3 & 4 & 5 & 6 \\
  1 & 2 & 3 & 4 & 5 & 6 \\
  \end{pmatrix}
\end{align*}

which is just the identity permutation.  So $1\mapsto 1$.  Continuing likewise,

\begin{align*}
  2 = (1\ 2) \mapsto
  \begin{pmatrix}
  1 & (1\ 2) & (2\ 3) & (1\ 3) & (1\ 2\ 3) & (1\ 3\ 2) \\
  (1\ 2) & 1 & (1\ 2\ 3) & (1\ 3\ 2) & (2\ 3) & (1\ 3)
  \end{pmatrix} = (1\ 2)(3\ 5)(4\ 6) \\\\
    3 = (2\ 3) \mapsto
    \begin{pmatrix}
    1 & (1\ 2) & (2\ 3) & (1\ 3) & (1\ 2\ 3) & (1\ 3\ 2) \\
    (2\ 3) & (1\ 3\ 2) & 1 & (1\ 2\ 3) & (1\ 3) & (1\ 2)
    \end{pmatrix} = (1\ 3)(2\ 6)(4\ 5) \\\\
      4 = (1\ 3) \mapsto
      \begin{pmatrix}
      1 & (1\ 2) & (2\ 3) & (1\ 3) & (1\ 2\ 3) & (1\ 3\ 2) \\
      (1\ 3) & (1\ 2\ 3) & (1\ 3\ 2) & 1 & (1\ 2) & (2\ 3)
      \end{pmatrix} = (1\ 4)(2\ 5)(3\ 6) \\\\
        5 = (1\ 2\ 3) \mapsto
        \begin{pmatrix}
        1 & (1\ 2) & (2\ 3) & (1\ 3) & (1\ 2\ 3) & (1\ 3\ 2) \\
        (1\ 2\ 3) & (1\ 3) & (1\ 2) & (2\ 3) & (1\ 3\ 2) & 1
        \end{pmatrix} = (1\ 5\ 6)(2\ 4\ 3) \\\\
          6 = (1\ 3\ 2) \mapsto
          \begin{pmatrix}
          1 & (1\ 2) & (2\ 3) & (1\ 3) & (1\ 2\ 3) & (1\ 3\ 2) \\
          (1\ 3\ 2) & (2\ 3) & (1\ 3) & (1\ 2) & 1 & (1\ 2\ 3)
          \end{pmatrix} = (1\ 6\ 5)(2\ 3\ 4) \\\\
\end{align*}



\question(4.2.8)\\
Prove that if $H$ has finite index $n$ then there is a normal subgroup $K$ of $G$ with $K\leq H$ and $|G:K|\leq n!$.

\begin{proof}
  Let $G/H$ be the set of cosets of $H$, and define the group action on this $G$-set of left-multiplication,

  \begin{align*}
    g\cdot (g'H) = (gg')H
  \end{align*}

  If we label the cosets $H_1, \dots, H_n$ then we define the following homomorphism $\varphi: G\rightarrow S_n$. We map $\varphi(g) = \sigma$ where $\sigma(i)=j$ if according to the action, $g\cdot H_i = H_j$.  As the kernel of a homomorphism, $\ker\varphi$ is a normal subgroup of $G$ and is contained in $H$.  So we take $K=\ker\varphi$.  Moreover by the first isomorphism theorem, $G/K$ is isomorphic to the image of $\varphi$ which is a subgroup of $S_n$ and therefore has order bounded by $n!$.  Then $|G/K|=|G:K|\leq n!$.

\end{proof}

\question(4.2.9)\\
Prove that if $p$ is prime and $G$ is a group of order $p^\alpha$ for some $\alpha\in\mathbb Z^+$ then every subgroup of index $p$ is normal in $G$.  Deduce that every group of order $p^2$ has a normal subgroup of order $p$.

\begin{proof}
  $p$ is the smallest prime dividing the order of the group, hence by corollary 5 the subgroup is normal.

  Now consider any non-identity element in a group of order $p^2$.  It either generates the whole group or a subgroup of order $p$.  If this element generates all of $G$ then $G$ is cyclic and hence abelian, so every subgroup is normal.  And moreover, since $G$ is cyclic, then for any divisor of the order of $G$ there is a subgroup of order that divisor.  So there is a normal subgroup of order $p$.

  On the other hand if the chosen element has order $p$ then the subgroup it generates satisfies the conditions of the first claim in this exercise.  Hence it is a normal subgroup of order $p$.
\end{proof}

\question(4.2.13)\\
Prove that if $|G|=2k$ where $k$ is odd then $G$ has a subgroup of index 2.  [Use Cauchy's Theorem to produce an element of order 2 and then use the preceding two exercises.]

\begin{proof}
  By Cauchy's Theorem, since 2 is prime and divides $|G|$, then there must be an element of order 2.  By the previous exercises, we take this element to be $x$ and since $|G|/|x|=k$ which is odd, then $\pi(x)$ is an odd permutation.  Here $\pi$ is the left regular permutation representation.  Now again by the previous exercises since $\pi(G)$ has an odd element then $G$ has a subgroup of index 2.
\end{proof}

\question(4.3.2(b))\\
Find all the conjugacy classes and their sizes in $Q_8$.

\begin{proof}
  $\{1\},\{-1\},\{i,-i\},\{j,-j\},\{k,-k\}$ with apparent sizes 1, 1, 2, 2, 2 which makes sense since every one of these both divides the order of the group and sums to the order of the group---which is what the class equation requires.
\end{proof}

\question(4.3.5)\\
If the center of $G$ is of index $n$, prove that every conjugacy class has at most $n$ elements.

\begin{proof}
  We know that in general for any element $g\in G$ that $Z(G)\leq C_G(g)\leq G$ and therefore since the index is finite $|G:Z(G)|=|G:C_G(g)|\cdot|C_G(g):Z(G)|$.  Each of these must be positive integers and

  \begin{align*}
    |G:C_G(g)| = n/|C_G(g):Z(G)| \leq n
  \end{align*}
\end{proof}

\question(4.3.20)\\
Let $\sigma\in A_n$.  Show that all elements in the conjugacy class of $\sigma$ in $S_n$ are conjugate in $A_n$ if and only if $\sigma$ commutes with an odd permutation.  [Use the preceding exercise.  This says that if $H$ is a normal subgroup of $G$ and $\mathcal K$ is a conjugacy class of $G$ contained in $H$ and $x\in\mathcal K$.  Then $\mathcal K$ is a union of $k$ conjugacy classes of equal size in $H$ where $k=|G:HC_G(x)|$.  Therefore a conjugacy class in $S_n$ which consists of even permutations is either a single conjugacy class under the action of $A_n$ or is a union of two classes of the same size in $A_n$.]

\begin{proof}
  Suppose $\sigma \in A_n$.  Suppose that every $\tau$ in the conjugacy class of $\sigma$ in $S_n$ is conjugate in $A_n$.  Then in particular $(1\ 2)\sigma(1\ 2)=\tau$ must be conjugate to $\sigma$ in $A_n$.  Hence $\tau = h\sigma h^{-1}$ with $h\in A_n$ and therefore

  \begin{align*}
    (1\ 2)\sigma (1\ 2) &= h\sigma h^{-1} \quad \Rightarrow \\\\
    \sigma(1\ 2)h &= (1\ 2)h\sigma
  \end{align*}

  where $(1\ 2)h$ is odd.

  For the converse suppose that $\sigma$ commutes with $g$ which is odd.  Now let $\tau$ be conujugate to $\sigma$ in $S_n$ so that there exists some $h\in S_n$ such that $h\sigma h^{-1}=\tau$.  We would like to find some even $h'$ such that $h'\sigma h'^{-1}=\tau$.
\end{proof}

\question(4.3.22)\\
Show that if $n$ is odd then the set of all $n$-cycles consists of two conjugacy classes of equal size in $A_n$.

\begin{proof}
  We use the previous exercise (21) which establishes that if a conjugacy class's elements have cycle types only of distinct odd integers, then it consists of two conjugacy classes.  Since the conjugacy class of an $n$-cycle only has $n$-cycles, then the cycle type is just $n$ which is a collection of distinct odd integers.  Hence this set consists of two conjugacy classes.

  From problem 19, we can infer that the conjugacy classes are of equal size.  This is because $n$ is odd and therefore the $n$-cycles are even permutations.
\end{proof}

\question(4.3.34)\\
Prove that if $p$ is a prime and $P$ is a subgroup of $S_p$ of order $p$, then $|N_{S_p}(P)|=p(p-1)$.  [Argue that every conjugate of $P$ contains exactly $p-1$ different $p$-cycles and use the formula for the number of $p$-cycles to compute the index of $N_{S_p}(P)$ in $S_p$.]

\begin{proof}
  Since the order of $P$ is prime then every non-identity element has order a $p$.  But a permutation has order a prime if and only if it is a $p$-cycle, so $P$ consists of $p-1$ cycles.  Any conjugate of a $p$-cycle has the same cycle type hence is a $p$-cycle, so for any $\sigma\in S_p$, the set $\sigma P\sigma^{-1}$ consists of $p-1$ distinct $p$-cycles. (Distinctness follows from the fact that cosets partition a group into equal-size partitions.)

  Since the number of $p$-cycles is given by

  \begin{align*}
    \frac{p\cdot(p-1)\cdots (p-p+1)}{p} = (p-1)!
  \end{align*}

  then the number of conjugates of $P$ is $\frac{(p-1)!}{p-1}=(p-2)!$. Then we have $|S_p:N_{S_p}(P)| = (p-2)!$.  Hence

  \begin{align*}
    p!/|N_{S_p}(P)| &= (p-2)! \quad \Longrightarrow \\\\
    |N_{S_p}(P)|&=p(p-1)
  \end{align*}
\end{proof}

\question(4.4.2)\\
Prove that if $G$ is abelian group of order $pq$ where $p$ and $q$ are distinct primes, then $G$ is cyclic. [Use Cauchy's Theorem to produce elements of order $p$ and $q$ and consider the order of their product.]

\begin{proof}
  By Cauchy's theorem there is an element of order $p$, call it $x$, and an element of order $q$, call it $y$.  Since $G$ is abelian the order of $xy$ is the least common multiple of $p,q$ which is $pq$ since these are prime.  Then $|xy|=pq$ and so $G$ is cyclic.
\end{proof}

\question(4.4.7)\\
If $H$ is the unique subgroup of a given order in a group $G$ prove $H$ is characteristic in $G$.

\begin{proof}
  For any automorphism of $G$, call it $\varphi$, the image $\varphi(H)$ is a subgroup of $G$ of order $|H|$ and hence $\varphi(H)=H$.  This shows $H$ is characteristic.
\end{proof}

\question(4.4.10)\\
Let $G$ be a group, let $A$ be an abelian normal subgroup of $G$, and write $\overline G = G/A$.  Show that $\overline G$ acts (on the left) by conjugation on $A$ by $\overline g\cdot a = gag^{-1}$, where $g$ is any representative of the coset $\overline g$ (in particular, show that this action is well defined).  Give an explicit example to show that this action is not well defined if $A$ is non-abelian.

\begin{proof}
  First we show that the action is well-defined, since it is given in terms of representatives of cosets.  If $a,b\in G$ such that $\overline a = \overline b$, and if $x\in A$, then first notice that $aA=bA$ so that there exists some $y\in A$ such that $a=by$.  Next, the penultimate equality below follows because $A$ is assumed to be an abelian subgroup.

  \begin{align*}
    \overline a \cdot x = axa^{-1} = byx(by)^{-1} = bxb^{-1}=\overline b\cdot x
  \end{align*}

  Since the action is well-defined, showing that it is a group action is straight-forward.  Since $A$ is the identity and $A\cdot x = (eA)\cdot x = exe^{-1}=x$ then the identity acts trivially.  If $a,b\in G$ then $a\cdot b\cdot x = abxb^{-1}a^{-1} = (ab)\cdot x$.

  \vspace{1cm}

  For an example of a non-abelian subgroup take $G=S_4$ and $A=\langle(1\ 2), (1\ 2\ 3)\rangle$.  Then $(1\ 4)A\cdot (1\ 4) = (1\ 4)(1\ 4)(1\ 4) = (1\ 4)$.  However, $(1\ 4)(1\ 2)=(1\ 2\ 4)\in (1\ 4)A$ and yet $(1\ 2\ 4)A \cdot (1\ 4) = (1\ 2\ 4)(1\ 4)(4\ 2\ 1) = (2\ 1)$.  Hence the action is not well-defined.
\end{proof}

\question(4.4.12)\\
Let $G$ be a group of order 3825.  Prove that if $H$ is a normal subgroup of order 17 then $H\leq Z(G)$.

\begin{proof}
  (Skipped)
\end{proof}

\question(4.5.1)\\
Let $G$ be a finite group and $p$ a prime.

Prove that if $P\in Syl_p(G)$ and $H$ is a subgroup of $G$ containing $P$ then $P\in Syl_p(H)$.  Give an example to show that, in general, a Sylow $p$-subgroup of a subgroup of $G$ need not be a Sylow $p$-subgroup of $G$.

\begin{proof}
  Since $P\leq H$ then we must have that $|P|$ divides $|H|$.  Since $|P|=p^\alpha$ then we must have $|H|=p^{\alpha+k}m$ for some $k\geq 0$ and $p\nmid m$, since it is always possible to group all factors of $p$ into a single factor $p^{\alpha+k}$.  This makes $P$ a Sylow $p$-subgroup of $H$.

  \vspace{1cm}

  With the group $G=S_4$ and the subgroup $H=\langle (1\ 2)\rangle$ we have that the subgroup $P=H$ is a Sylow 2-subgroup of $H$.  However, because $|S_4|=24=2^3\cdot 3$ then a Sylow 2-subgroup of $S_4$ would have order 8.  Hence $P$ cannot be a Sylow 2-subgroup of $S_4$.
\end{proof}

\question(4.5.5)\\
Show that a Sylow $p$-subgroup of $D_{2n}$ is cyclic and normal for every odd prime $p$.

\begin{proof}
  Suppose $P$ is a Sylow $p$-subgroup, so that $|P|=p^\alpha$.  Then $P$ cannot have any element of even order and cannot have any reflection.  Hence $P$ is a subgroup of the group of rotations $R\leq D_{2n}$, hence $P$ is cyclic.  Therefore $P$ is the unique subgroup of $R$ of order $p^\alpha$ and hence the unique such subgroup of $D_{2n}$.  So $P$ is normal in $D_{2n}$.
\end{proof}

\question(4.5.13)\\
Prove that a group of order 56 has a normal Sylow $p$-subgroup for some prime $p$ dividing its order.

\begin{proof}
  Since $56=2^3\cdot 7$ then

  \begin{align*}
    n_2 | 7 \\\\
    n_7 | 8 \\\\
  \end{align*}

  therefore $n_2\in \{1,7\}$ and $n_7\in \{1,2,4,8\}$.  Moreover $n_7\equiv 1\mod 7$ so then we only have $n_7\in\{1,8\}$.  If $n_2=1$ then already the Sylow 2-subgroup is normal so suppose $n_2=7$.  Then there are $7\cdot 7=49$ non-identity elements in these 7 different Sylow 2-subgroup, leaving $56-49=7$ elements remaining.  Hence there can only be one Sylow 7-subgroup and therefore this is one is normal.
\end{proof}

\question(4.5.23)\\
Prove that if $|G|=462$ then $G$ is not simple.

\begin{proof}
  (Skipped)
\end{proof}

\question(4.5.32)\\
Let $P$ be a Sylow $p$-subgroup of $H$ and let $H$ be a subgroup of $K$.  If $P\trianglelefteq H$ and $H\trianglelefteq K$, prove that $P$ is normal in $K$.  Deduce that if $P\in Syl_p(G)$ and $H=N_G(P)$, then $N_G(H)=H$.

\begin{proof}
  (Skipped)
\end{proof}

\question(4.5.44)\\
Let $p$ be the smallest prime dividing the order of the finite group $G$.  If $P\in Syl_p(G)$ and $P$ is cyclic prove that $N_G(P) = C_G(P)$.

\begin{proof}
  (Skipped)
\end{proof}



\end{questions}



















\end{document}
