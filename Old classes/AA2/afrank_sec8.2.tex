\documentclass[11pt]{article}

\usepackage{epsfig}
\usepackage{float}
\usepackage{amsmath}
\usepackage{amssymb}
\usepackage{mathrsfs}
\usepackage{color}
\usepackage{paralist}
%\usepackage{enumitem}
\usepackage{array}
\usepackage{enumerate}
\usepackage{wasysym}

\pagestyle{headings} \setlength{\textwidth}{7.5in}
\setlength{\textheight}{10.0in} \setlength{\topmargin}{-.5in}
%\setlength{\headheight}{0in}
%\setlength{\headsep}{0in}
%\setlength{\topskip}{0in}
%\setlength{\voffset}{0in} \setlength{\oddsidemargin}{-.5in} \setlength{\evensidemargin}{0in}
%\setlength{\unitlength}{.5in}


\newcommand{\st}{\rule[-0.25in]{0in}{0.5in}}
\newcommand{\stt}{\rule[-0.5in]{0in}{1in}}
\newcommand{\yn}{\makebox[1in]{\st\ YES} \makebox[1in]{\st\ NO}}
\newcommand{\bl}{\framebox[1in]{\st \hfill}}
\newcommand{\bll}{\framebox[2in]{\st \hfill}}
\newcommand{\blll}{\framebox[3.5in]{\st \hfill}}
\newcommand{\U}{\vspace{0.8in}}
\newcommand{\V}{\vspace{1in}}
\newcommand{\W}{\vspace{1.5in}}
\newcommand{\R}{\rule{0in}{0.25in}\hrulefill\rule{0in}{0.25in}}
\newcommand{\DS}{\displaystyle}
\setlength{\textwidth}{7in} \setlength{\textheight}{10in}
\setlength{\evensidemargin}{-.5in} \setlength{\oddsidemargin}{-0.5in}
\setlength{\topmargin}{-1in} \setlength{\parindent}{24pt}
\def\noin{\noindent}
\def\ds{\displaystyle}
\newcommand{\mysec}[1]{\centerline{\normalfont\Large\bfseries #1} \bigskip}
\newenvironment{changemargin}[2]{%
\begin{list}{}{%
\setlength{\topsep}{0pt}%
\setlength{\leftmargin}{#1}%
\setlength{\rightmargin}{#2}%
\setlength{\listparindent}{\parindent}%
\setlength{\itemindent}{\parindent}%
\setlength{\parsep}{\parskip}%
}%
\item[]}{\end{list}}

\begin{document}
\thispagestyle{empty}
%
%

%\hspace{5.25in} \textbf{Version A}\\[-26pt]

\large
\noin \textbf{MA 638 - Section 8.2 Homework}\\


\normalsize

1. Let $R$ be  an  integral  domain  and  suppose  that  every  prime  ideal  in $R$ is  principal.   This  exercise proves that every ideal of $R$ is principal, i.e., $R$ is a PID.

(a)  Assume that the set of ideals of $R$ that are not principal is nonempty and prove that this set has a maximal element under inclusion (which, by hypothesis, is not prime).  [Hint:  Use Zorn's Lemma.] 

(b)  Let $I$ be an ideal which is maximal with respect to being nonprincipal, and let $a,b\in R$ with $ab \in I$ but $a\not\in I$ and $b\not\in I$.  Let $I_a = (I,a)$ be the ideal generated by $I$ and $a$, let $I_b= (I,b)$ be the ideal generated by $I$ and $b$, and define $J=\{r \in R|rI_a \subseteq I\}$.  Prove that $I_a= (\alpha )$ and $J= (\beta)$ are principal ideals in $R$ with $I\subseteq I_b \subseteq J$ and $I_aJ= (\alpha\beta)\subseteq I$.

(c)  If $x\in I$ show that $x=s\alpha$ for some $s\in J$.  Deduce that $I=I_aJ$ is principal, a contradiction, and conclude that $R$ is a PID.

{\it Proof:} (a)  Let $\mathscr I\ne \emptyset$ be the set of all ideals of $R$ which are not principal, and let $I_1\in \mathscr I$.  If $I_1$ is not maximal for $\mathscr I$ then there is some $I_2\in\mathscr I$ such that $I_1\subsetneq I_2$.  Proceeding recursively, we develop an increasing chain of subsets 

\begin{align*}
    I_1\subsetneq I_2 \subsetneq \dots 
\end{align*}

with $I_n\in \mathscr I$ for $n=1,2,\dots $  Sets are of course partially ordered by set containment.  Now set $I=\displaystyle \bigcup_{i=1}^\infty I_i$.  If $r\in R$ and $ri\in rI$ then for some $n$ we have $i\in I_n$.  Since $I_n$ is an ideal therefore $ri\in I_n\subseteq I$ so that $I$ is an ideal.  Moreover, if $I$ were prinicpal and $I=(\gamma)$ then we would have $\gamma\in I_n$ for some $n$, in which case $I=(\gamma)\subseteq I_n$.  Since $I_n\subseteq I$ then this would make $I=I_n$ and then $I_n$ is principal, contrary to assumption.  Hence $I$ is not principal.  

(b) Since $I$ is maximal and $I\subsetneq I_a$, then $I_a\not\in \mathscr I$ and therefore $I_a$ is principal.  We let $\alpha$ be that element which generates $I_a$.  

We want to also show that $I\subsetneq J$, and since we clearly have $I\subsetneq I_b$, then it suffices to show $I_b\subseteq J$.  Now if $ci_a+db \in I_b$ then we seek to show that $(ci_a+db)I_a\subseteq I$.  So also let $ei_b+fa\in I_a$, and therefore 

\begin{align*}
    (ci_a+db)(ei_b+fa) = cei_ai_b + cfai_a+dbei_b + dbfa.
\end{align*}

Since $I$ is an ideal then $cei_ai_b, cfai_a, dbei_b\in I$. We also know that $ab\in I$ and therefore $dfab\in I$.  Hence $(ci+db)(ei+fa)=cei_ai_b + cfai_a+dbei_b + dbfa\in I$, hence $(ci+db)\in J$, hence $I_b\subseteq J$.  From this it follows that 

\begin{align*}
    I\subsetneq I_b \subseteq J
\end{align*}

and from that we infer that $J$ is principal.  Call the element which generates it $\beta$, so that $(\beta) = J$.  

All that then remains is to show that $I_aJ = (\alpha\beta)\subseteq I$.  In general we have that $(\alpha)(\beta) = (\alpha\beta)$ so that this is trivial. To complete this part, let $j\in J$ so that we have $jI_a \subseteq I$. Therefore if $s\in I_a$ then we have $js\in I$.  But this shows that for an arbitrary $j\in J$ and $s\in I_a$ we have $sj\in I$ and hence $I_aJ\subseteq I$.  

(c) Oh man, finally!  Suppose $x\in I$ and therefore $x\in I_a$, so that we must have some $s\in R$ such that $x=s\alpha$.  It's easy to see that $s\cdot (\alpha) = (s\alpha)$ since an arbitrary element from the left is $s\cdot (c\cdot \alpha)$ for some $c\in R$, but this is the same as an arbitrary element from the right, $c\cdot (s\cdot \alpha)$.  Therefore  

\begin{align*}
    sI_a = s(\alpha) = (s\alpha)=(x)\subseteq I.
\end{align*}

Then $sI_a \subseteq I$ and therefore $s\in J$.  Since we now have $I\subseteq (\alpha\beta)$ and $(\alpha\beta)\subseteq I$ then we have $I\subseteq I_aJ=(\alpha\beta)$ and therefore $I$ is principal.  But this contradicts our finding from part (a) that $I$ is not principal.  

Hence the assumption that the set of non-principal ideals is empty must be false.  Therefore $R$ is a PID.  

\end{document}
