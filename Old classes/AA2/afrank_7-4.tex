\documentclass[11pt]{article}

\usepackage{epsfig}
\usepackage{float}
\usepackage{amsmath}
\usepackage{amssymb}
\usepackage{color}
\usepackage{paralist}
%\usepackage{enumitem}
\usepackage{array}
\usepackage{enumerate}

\pagestyle{headings} \setlength{\textwidth}{7.5in}
\setlength{\textheight}{10.0in} \setlength{\topmargin}{-.5in}
%\setlength{\headheight}{0in}
%\setlength{\headsep}{0in}
%\setlength{\topskip}{0in}
%\setlength{\voffset}{0in} \setlength{\oddsidemargin}{-.5in} \setlength{\evensidemargin}{0in}
%\setlength{\unitlength}{.5in}


\newcommand{\st}{\rule[-0.25in]{0in}{0.5in}}
\newcommand{\stt}{\rule[-0.5in]{0in}{1in}}
\newcommand{\yn}{\makebox[1in]{\st\ YES} \makebox[1in]{\st\ NO}}
\newcommand{\bl}{\framebox[1in]{\st \hfill}}
\newcommand{\bll}{\framebox[2in]{\st \hfill}}
\newcommand{\blll}{\framebox[3.5in]{\st \hfill}}
\newcommand{\U}{\vspace{0.8in}}
\newcommand{\V}{\vspace{1in}}
\newcommand{\W}{\vspace{1.5in}}
\newcommand{\R}{\rule{0in}{0.25in}\hrulefill\rule{0in}{0.25in}}
\newcommand{\DS}{\displaystyle}
\setlength{\textwidth}{7in} \setlength{\textheight}{10in}
\setlength{\evensidemargin}{-.5in} \setlength{\oddsidemargin}{-0.5in}
\setlength{\topmargin}{-1in} \setlength{\parindent}{24pt}
\def\noin{\noindent}
\def\ds{\displaystyle}
\newcommand{\mysec}[1]{\centerline{\normalfont\Large\bfseries #1} \bigskip}
\newenvironment{changemargin}[2]{%
\begin{list}{}{%
\setlength{\topsep}{0pt}%
\setlength{\leftmargin}{#1}%
\setlength{\rightmargin}{#2}%
\setlength{\listparindent}{\parindent}%
\setlength{\itemindent}{\parindent}%
\setlength{\parsep}{\parskip}%
}%
\item[]}{\end{list}}

\begin{document}
\thispagestyle{empty}
%
%

%\hspace{5.25in} \textbf{Version A}\\[-26pt]

\large
\noin \textbf{MA 638 - Section 7.4 Homework}\\


\normalsize



Throughout assume $R$ is a ring with identity $1\ne 0$.

\begin{enumerate}


\item Let $R$ be a commutative ring.  Prove that the principal ideal generated by $x$ in $R[x]$ is a prime ideal if and only if $R$ is an integral domain.  Prove that $(x)$ is a maximal ideal if and only if $R$ is a field.

For each part, we will use the following:  Consider the evaluation map $\varphi: R[x]\to R$.  If $p(x)\in R[x]$ then $\varphi$ is given by $\varphi(p(x))=p(0)$. Obviously we have both $\ker \varphi = (x)$ and $\text{Im} \varphi = R$.  Now $R[x]/(x) \cong R$ by the first isomorphism theorem.

{\it Proof part 1:}  Suppose $(x)$ in $R[x]$ is prime.  By theorem 13 of section 7.4, since $(x)$ is prime therefore $R[x]/(x)$ is an integral domain.  Since $R[x]/(x)\cong R$, then $R$ is an integral domain.

For the converse suppose that $R$ is an integral domain.  Therefore $(x)$ is prime, again by theorem 13.

{\it Proof part 2:} Suppose $(x)$ is maximal and therefore by theorem 12 of section 7.4 $R[x]/(x)$ is a field.  Since $R[x]/(x)\cong R$ then we have that $R$ is a field.  For the converse, suppose $R$ is a field, then by theorem 12, $(x)$ is maximal.

\item Assume $R$ is commutative.  Prove that if $P$ is a prime ideal of $R$ and $P$ contains no zero divisors, then $R$ is an integral domain.

{\it Proof:} Let $r,s\in R$ such that $rs=0$.  Since $P$ is an ideal, then $rs=0\in P$.  Since $P$ is prime either $r\in P$ or $s\in P$.  Without loss of generality suppose $r\in P$.  Therefore $r$ is not a zero divisor, and since $rs=0$ then either $r=0$ or $s=0$.  Hence $R$ is an integral domain.

\item Let $\phi: R\to S$ be a homomorphism of commutative rings.  Prove that if $P$ is a prime ideal of $S$ then either $\phi^{-1}(P) = R$ or $\phi^{-1}(P)$ is a prime ideal of $R$.  Apply this to a special case when $R$ is a subring of $S$ and $\phi$ is the inclusion homomorphism to deduce that if $P$ is a prime ideal of $S$, then $P\cap R$ is either $R$ or a prime ideal of $R$.

{\it Proof:} First we establish that $\phi^{-1}(P)$ is an ideal, beginning with the fact that it is a commutative subgroup under addition.  If $a,b\in \phi^{-1}(P)$ then $\phi(a),\phi(b)\in P$ and since $P$ is an ideal, $\phi(a)+\phi(b)\in P$.  Since $\phi$ is a homomorphism $\phi(a+b)=\phi(a)+\phi(b)$ and therefore $a+b\in \phi^{-1}(P)$.  Moreover $-\phi(a)=\phi(-a)\in P$ and so $-a\in \phi^{-1}(P)$.  Finally, associativity and commutativity follow from the fact that $R$ is a ring.

To next show closure under arbitrary products, let $r\in R$.  Then $\phi(r)\phi(a)\in P$ since $P$ is an ideal.  But then $\phi(ra)=\phi(r)\phi(a)$ and so $ra\in\phi^{-1}(P)$.  All that then remains is to show associativity and distributivity, but these are immediate from the fact that $R$ is a ring.

Now that we know $\phi^{-1}(P)$ is an ideal, let $a,b\in R$ and $ab\in \phi^{-1}(P)$.  Therefore there exists some $p\in P$ such that $\phi(ab)=p$.  Since $\phi$ was assumed to be a homomorphism then $\phi(ab)=\phi(a)\phi(b)\in P$ and therefore either $\phi(a)\in P$ or $\phi(b)\in P$.  Without loss of generality suppose $\phi(a)\in P$ and therefore $a\in\phi^{-1}(P)$.  This shows that if $\phi^{-1}(P)\ne R$ then $\phi^{-1}(P)$ is prime, which is equivalent to the statement that $\phi^{-1}(P)=R$ or $\phi^{-1}(P)$ is prime.

If $R$ is a subring of $S$ and $\phi$ is the inclusion homomorphism, and if $P$ is a prime ideal of $S$, then we have that $\phi^{-1}(P)$ is prime.  But because $\phi$ is the inclusion homomorphism then $\phi^{-1}(P)$ just is those elements of $R$ which are elements in $P$.  That is to say $\phi^{-1}(P)=P\cap R$.  Hence if $P\cap R\ne R$ then $P\cap R$ is prime.

\end{enumerate}
\end{document}
