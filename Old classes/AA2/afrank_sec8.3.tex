\documentclass[11pt]{article}

\usepackage{epsfig}
\usepackage{float}
\usepackage{amsmath}
\usepackage{amssymb}
\usepackage{mathrsfs}
\usepackage{color}
\usepackage{paralist}
%\usepackage{enumitem}
\usepackage{array}
\usepackage{enumerate}
\usepackage{wasysym}

\pagestyle{headings} \setlength{\textwidth}{7.5in}
\setlength{\textheight}{10.0in} \setlength{\topmargin}{-.5in}
%\setlength{\headheight}{0in}
%\setlength{\headsep}{0in}
%\setlength{\topskip}{0in}
%\setlength{\voffset}{0in} \setlength{\oddsidemargin}{-.5in} \setlength{\evensidemargin}{0in}
%\setlength{\unitlength}{.5in}


\newcommand{\st}{\rule[-0.25in]{0in}{0.5in}}
\newcommand{\stt}{\rule[-0.5in]{0in}{1in}}
\newcommand{\yn}{\makebox[1in]{\st\ YES} \makebox[1in]{\st\ NO}}
\newcommand{\bl}{\framebox[1in]{\st \hfill}}
\newcommand{\bll}{\framebox[2in]{\st \hfill}}
\newcommand{\blll}{\framebox[3.5in]{\st \hfill}}
\newcommand{\U}{\vspace{0.8in}}
\newcommand{\V}{\vspace{1in}}
\newcommand{\W}{\vspace{1.5in}}
\newcommand{\R}{\rule{0in}{0.25in}\hrulefill\rule{0in}{0.25in}}
\newcommand{\DS}{\displaystyle}
\setlength{\textwidth}{7in} \setlength{\textheight}{10in}
\setlength{\evensidemargin}{-.5in} \setlength{\oddsidemargin}{-0.5in}
\setlength{\topmargin}{-1in} \setlength{\parindent}{24pt}
\def\noin{\noindent}
\def\ds{\displaystyle}
\newcommand{\mysec}[1]{\centerline{\normalfont\Large\bfseries #1} \bigskip}
\newenvironment{changemargin}[2]{%
\begin{list}{}{%
\setlength{\topsep}{0pt}%
\setlength{\leftmargin}{#1}%
\setlength{\rightmargin}{#2}%
\setlength{\listparindent}{\parindent}%
\setlength{\itemindent}{\parindent}%
\setlength{\parsep}{\parskip}%
}%
\item[]}{\end{list}}

\begin{document}
\thispagestyle{empty}
%
%

%\hspace{5.25in} \textbf{Version A}\\[-26pt]

\large
\noin \textbf{MA 638 - Section 8.3 Homework}\\


\normalsize

1. Let $R=\mathbb Z[\sqrt{-n}]$ where $n$ is a squarefree integer greater than 3.

(a)  Prove that $2,\sqrt{-n},$ and $1 +\sqrt{-n}$ are irreducibles in $R$.

(b)  Prove that $R$ is not a UFD. Conclude that the quadratic integer ring $\mathcal O$ is not a UFD for $D\equiv 2, 3 \pmod{ 4}, \ D <-3$ (so also not an ED and not a PID). [Hint:  Show that either $\sqrt{-n}$ or $1 +\sqrt{-n}$ is not prime].

(c)  Give an explicit ideal in $R$ that is not principal.

(a) {\it Proof:}  Suppose $2=ab$ then $N(2) = 4 = N(a)N(b)$.  Then $N(a)$ can only be 1, 2, or 4.  If $N(a)=1$ then $a$ is a unit, and therefore there is nothing to prove.  If $N(a)=4$ then $N(b)=1$ so $b$ is a unit, and there is nothing to prove.  So consider $N(a)=2$.  If $a=x+y\sqrt{-n}\in \mathbb Z[\sqrt{-n}]$ then $2=x^2+ny^2$.  Now if $y\ne 0$ then $x^2+ny^2 \geq n > 3$ and therefore this case is impossible, so $y=0$. But then $x^2 = 2$ which has no solution in the integers, so this also is impossible.  Hence $N(a)\ne 2$ and so 2 is irreducible.

Next suppose $\sqrt{-n}=ab$ so that $N(\sqrt{-n})=n=N(a)N(b)$.  If $a=x+y\sqrt{-n}$ and $b=w+z\sqrt{-n}$ then 

\begin{align*}
    N(a)N(b)&=(x^2+ny^2)(w^2+nz^2) \\\\
    &=n\\\\
    &= (xw)^2+n( [xz]^2+[yw]^2) + n^2(yz)^2.
\end{align*}

Now if $yz>0$ then $N(a)N(b) \geq n^2 > n$ which is impossible, so either $y$ or $z$ is zero.  Without loss of generality suppose $y=0$.  But then $a=x$ is just an integer, and then $ab=x(w+z\sqrt{-n})=\sqrt{-n}$.  Therefore $xw = 0$ and $xz = 1$.  The latter requires $x=\pm 1$ since no other number can divide 1 in $\mathbb Z$ and therefore $a$ is a unit.  So $\sqrt{-n}$ is irreducible.  

(b) {\it Proof:} Suppose that $D=-n$ is congruent to 2 mod 4.  We will show that $\sqrt{-n}$ is not prime in $\mathbb Z [\sqrt{-n}]$.  First note that in this case $n$ is even, since $-n=2+4k=2(1+2k)$ for some integer $k$.  So let $n=2m$.

We will use the fact that an element is prime if and only if it generates a prime ideal.  So it suffices to show that $(\sqrt{-n})$ is not prime.  First note that $n=(-\sqrt{-n})\sqrt{-n}$ so that $2m=n\in (\sqrt{-n})$.  However, $2\not\in(\sqrt{-n})$ since if it were, $2=a\sqrt{-n}$ for some $a$ but then $N(2)=4=N(\sqrt{-n})N(a)=nN(a)$. However, $n > 3$ would entail that $N(a)=1$ and then $\sqrt{-n} = \pm 2$ which is impossible.  Also we have that $m\not\in (\sqrt{-n})$ for if $m=a\sqrt{-n}$ then $2m=n=2a\sqrt{-n}$.  From this it would follow that $n^2 = 2an$ and so $n^2 - 2an = 0$ so that 

\begin{align*}
    n(n-2a)&=0 \quad \Rightarrow \\\\
    n&=2a=2m \quad \Rightarrow \\\\
    a&=m
\end{align*}

But then $a=a\sqrt{-n}$ so that $n=1$ which contradicts $n > 3$.  Hence $(\sqrt{-n})$ is not a prime ideal and therefore $\sqrt{-n}$ is not prime.  Since $\mathcal O$ then has some irreducible element that is not prime, it must not be a UFD.

Next consider $D=-n$ is congruent to 3 mod 4 so that $-n = 3+4k$ for some $k$.  Then we show that $1+\sqrt{-n}$ is not prime.  First note that because $(1+\sqrt{-n})(1-\sqrt{-n})=1-n\in (1+\sqrt{-n})$ then therefore $1+3+4k = 4(k+1)\in (1+\sqrt{-n})$. However $4\not\in (1+\sqrt{-n})$ because if it were and $4=a(1+\sqrt{-n})$ then we would have $N(4)=16=N(a)(1+n)$.  Now since $n>3$ and since $n$ is squarefree, then also $n\ne 4$ so that in fact $n\geq 5$.  For $N(a)(1+n)$ to have exactly 4 factors of 2, we must have $1+n$ equal to either 8 or 16.  If $1+n=8$ so that $n=7$ then $4=a(1+\sqrt{-7})$.  If we call $a=x+y\sqrt{-7}$ then 

\begin{align*}
    4&=(x+y\sqrt{-7})(1+\sqrt{-7}) = (x-7y)+(x+y)\sqrt{-7} \quad \Rightarrow \\\\
    4&=x-7y \quad \text{and}\\
    0&=x+y
\end{align*}

But this implies $x=-y$ and therefore $4=-y-7y=-8y$ which is not true for any integer $y$.  

On the other hand we also have $k+1\not\in (1+\sqrt{-n})$ since then $4(k+1)=1-n=4a(1+\sqrt{-n})$ so that $

\begin{align*}
    
\end{align*}

\end{document}
