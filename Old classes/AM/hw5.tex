\documentclass[12pt]{article}
\usepackage{fullpage, amsfonts, amsmath, amsthm, amssymb, array, enumerate, systeme, wasysym, mathrsfs}
\usepackage{geometry}


\begin{document}
\pagestyle{empty}

 {\noindent \textbf{\large MA 630 - Homework 5 (Module 3 - Section 1)}
\vspace{.25in}

\noindent Solutions must be typeset in \LaTeX \ and submitted to Canvas as a .pdf file. When applicable, write in complete sentences. Use only results which have been discussed in our class.
\vspace{.25in}

\begin{enumerate}
\item Let $n \in \mathbb{Z}$. Prove that $\gcd(5n + 2, 12n + 5) = 1$.

Since we have that $(-12)(5n+2) + (5)(12n+5) = 1$ then setting $a=5n+12$ and $b=12n+5$ we have that

\begin{align*}
  1\in \{ax+by| x,y\in\mathbb Z\}.
\end{align*}

There is no natural number less than 1, so 1 is the least such natural number.  Hence by theorem 3.10 we have that $\gcd(a,b)=1$.  \qed

\item Let $a$ and $b$ be integers which are not both zero, and let $d = \gcd(a,b)$. Prove that $\gcd\left(\dfrac{a}{d}, \dfrac{b}{d}\right)  = 1$.

{\it Proof:} By Theorem 3.10 we know there are $m,n\in\mathbb Z$ such that $am+bn=d$.  Hence $\left(\frac{a}{d}\right)m+\left(\frac{b}{d}\right)n = 1$ and therefore

\begin{align*}
  1\in\left\{\left(\frac{a}{d}\right)m+\left(\frac{b}{d}\right)n|m,n\in\mathbb Z\right\}.
\end{align*}

Since there is no natural number less than 1, then 1 is the least natural number in the set.  We also know that $\frac{a}{d}$ is a natural number since $d$ divides $a$ and therefore $a = dp$ for some natural number $p$.  Hence $\frac a d = p$ which is a natural number.  By the same argument, $\frac{b}{d}$ is a natural number.  Hence by theorem 3.10 it follows that  $gcd\left(\frac{a}{d},\frac{b}{d}\right)=1$. \qed

\item Let $a, b, q, r \in \mathbb{Z}$ with $b \neq 0$. Prove that if $a = bq + r$, then $\gcd(a,b) = \gcd(b,r)$. \emph{Hint: Let $d = \gcd(a,b)$. Use Theorem 3.9 to characterize $\gcd(b,r)$, and then show that $d = \gcd(b,r)$.}

{\it Proof:} Suppose $a=bq+r$ with $b\ne 0$.  By definition $\gcd(b,r)$ divides both $b$ and $r$.  So let $m,n$ be integers such that $b = m\gcd(b,r)$ and $r=n\gcd(b,r)$.  Then since

\begin{align*}
  a = m\gcd(b,r)q+n\gcd(b,r) = \gcd(b,r)(mq+n)
\end{align*}

we now have that $\gcd (b,r)$ divides $a$.  Since we have already noted that $\gcd(b,r)$ divides $b$, then $\gcd(b,r)$ satisfies the first property of theorem 3.9 applied to $d= gcd(a,b)$.

Next let $c$ be any integer which divides $a$ and $b$.  Let $a = cf$ and $b=cg$.  We want to show that $c$ divides $\gcd(b,r)$, in order to demonstrate the second property of 3.9. Now by theorem 3.10 we have that $\gcd(b,r) = bx+ry$ for some integers $x,y$.  Moreover, since $r=a-bq$ then

\begin{align*}
  \gcd(b,r) = cgx + (a-bq)y = cgx+cf-cgqy = c(gx+f-gqy).
\end{align*}

This shows that $c$ divides $\gcd(b,r)$ as desired.

Hence by theorem 3.9 $\gcd(b,r)= \gcd(a,b)$.

\item Use the Euclidean algorithm to find an integer $x$ such that $2314 x - 1$ is divisible by 3181.

We apply the Euclidean algorithm to 2314 and 3181:

\begin{align*}
  3181 &= 2314\cdot 1 + 867 & 867 &= 3181-2314\\
  2314 &= 867\cdot 2 + 580 & 580 &= 2314-867\cdot 2\\
  867 &= 580\cdot 1 + 287 & 287 &= 867-580\\
  580 &= 287\cdot 2 + 6 & 6 &= 580-287\cdot 2\\
  287 &= 6\cdot 47 + 5 & 5 &= 287 - 6\cdot 47\\
  6 &= 5\cdot 1 + 1 & 1 &= 6-5\\
  5 &= 1\cdot 5 + 0. &
\end{align*}

\begin{align*}
  1 &= 6-5 \\
  &= 6 - (287-6\cdot 47) \\
  &= 6\cdot 48 - 287 \\
  &= (580-287\cdot 2)\cdot 48-287 \\
  &= 287\cdot (-97)+580\cdot 48 \\
  &= (867-580)\cdot(-97)+580\cdot 48 \\
  &= 580\cdot(145)-867\cdot 97 \\
  &= (2314-867\cdot 2)\cdot 145 - 867\cdot 97 \\
  &= -867\cdot 387 + 2314\cdot 145 \\
  &= -(3181-2314)\cdot 387+2314\cdot 145 \\
  &= 2314\cdot 532 - 3181\cdot 387. \\
\end{align*}

This then shows that

\begin{align*}
  3181\cdot 387 = 2314\cdot 532-1
\end{align*}

which shows that 3181 divides $2314\cdot 532-1$, and so a choice for $x$ is 532.

\item Let $p\geq 5$ be a prime.
	\begin{enumerate}
	\item Use the division algorithm to prove that there exists $k \in \mathbb{Z}$ such that either $p = 6k + 1$ or $p = 6k - 1$.
	\item Prove that 24 divides $p^2 - 1$. \emph{Hint: The integer $k$ in part (a) is either even or odd.}
	\end{enumerate}

  {\it (a) Proof:} By the division algorithm there must exist some $q_1$ and $0\leq r_1<3$ such that $p=3q_1+r_1$.  Also there must exist some $q_2$ and $0\leq r_2<2$ such that $p=2q_2+r_2$. Also $r_1\ne 0$ otherwise we have that 3 divides $p$, which cannot be true since $p$ is prime and greater than 4.  Similarly $r_2\ne 0$, and in this case the only possibility then is $r_2=1$.  Hence $p=2q_2+1$.

  The remaining cases for $r_1$ are $r_1=1,2$.  If $r_1=1$ then

  \begin{align*}
    p=3q_1+1=2q_2+1
  \end{align*}

  and so $3q_1 = 2q_2$.  Since 3 is prime, and 3 does not divide 2, then by corollary 3.13 we have that 3 divides $q_2$, so write $q_2 = 3k$.  Then we have

  \begin{align*}
    p=2q_2+1=2(3k)+1 = 6k+1
  \end{align*}

  On the other hand if $r_1 = 2$ then

  \begin{align*}
    p = 3q_1+2 = 2q_2+1
  \end{align*}

  so $3q_1+1=2q_2$.  Since $2q_2$ is even, the quantity on the left must be as well.  But then 1 is odd, and an odd plus an even is odd.  So $3q_1$ cannot be even, and must be odd.  If $q_1$ were even then $3q_1$ would be even, so we must have $q_1$ is odd.  (Note: I'm hoping at this point in the course we can freely use facts like these.)  So let $q_1=2m+1$ for some integer $m$.  Therefore

  \begin{align*}
    p&=3q_1+2 \\
    &= 3(2m+1)+2 \\
    &= 6m+5 \\
    &= 6(m+1)-6+5 \\
    &= 6(m+1)-1.
  \end{align*}

  Setting $k=m+1$ we then have that $p=6k-1$.

  In both cases we have seen that either $p=6k+1$ or $p=6k-1$.  Since this exhausts all possible cases, the proof is complete.  \qed

  {\it (b) Proof:} We have that

  \begin{align*}
    p^2-1 = (p+1)(p-1).
  \end{align*}

  If $p=6k+1$ then

  \begin{align*}
    (p+1)(p-1) &= (6k+2)(6k) \\
    &= 12k(3k+1).
  \end{align*}

  In that case, either $k$ is even or odd.  If $k$ is even and $k=2m$ then

  \begin{align*}
    (p+1)(p-1) &= 12(2m)(3k+2) \\
    &=24m(3k+2)
  \end{align*}

  which shows that $p^2-1$ is divisible by 24.  On the other hand if $k$ is odd then let $k=2m+1$.  Therefore

  \begin{align*}
    (p+1)(p-1) &= 12k(3(2m+1)+1)\\
    &= 12k(6m+4) \\
    &= 24k(3m+2)
  \end{align*}

  which again shows that $p^2-1$ is divisible by 24.

  \vspace{1cm}

  Now we consider the case where $p=6k-1$.  Then

  \begin{align*}
    p^2-1 &= (p+1)(p-1) \\
    &= 6k(6k-2) \\
    &= 12k(3k-1).
  \end{align*}

  Now if $k$ is even and $k=2m$ then

  \begin{align*}
    p^2-1 = 24m(3k-1)
  \end{align*}

  and we have $p^2-1$ is divisible by 24.  If $k$ is odd and $k=2m+1$ then

  \begin{align*}
    p^2-1 &= 12k(3(2m+1)-1) \\
    &= 12k(6m+2) \\
    &= 24k(3m+1).
  \end{align*}

  Again this shows $p^2-1$ is divisible by 24.  Since this exhausts all possible cases, then we must have that $p^2-1$ is divisible by 24. \qed

  \end{enumerate}
\end{document}
