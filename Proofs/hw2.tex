\documentclass[12pt]{article}
\usepackage{fullpage, amsfonts, amsmath, amsthm, amssymb, array, enumerate, systeme, wasysym}
\usepackage{marvosym}

\begin{document}
\pagestyle{empty}

 {\noindent \textbf{\large MA 630 - Homework 2 (Module 1 - Sections 1 and 2)}
\vspace{.25in}

\noindent Solutions must be typeset in \LaTeX \ and submitted to Canvas as a .pdf file. When applicable, write in complete sentences.
\vspace{.25in}


\begin{enumerate}

\item Let $x$ be a positive real number. Prove that if $\displaystyle x - \frac{2}{x} > 1$, then $x > 2$ by
	\begin{enumerate}
	\item direct proof.
	\item contrapositive proof.
	\item proof by contradiction.
	\end{enumerate}
	
{\it Solution:} (a) Direct proof: We assume $x>0$ and $x - \frac 2 x > 1$, and try to show that $x > 2$.  Each inequation below implies the next:

\begin{align*}
    x - \frac 2 x &> 1 \\
    x^2 - 2 &> x \\
    x^2 - x - 2 &> 0 \\
    (x-2)(x+1) &> 0\\
\end{align*}

So $(x-2)(x+1)$ is a positive real.  There are three cases: either both are positive, or both are negative, or one is positive and the other negative.  This last case is impossible since the product of a positive and negative is negative. But also both cannot be negative because $x+1<0$ entails $x<-1<0$ which contradicts that $x$ is positive, as in the assumption.  Hence both factors are positive, by disjunction elimination.  So $x-2>0$ and so $x>2$.  \qed

(b) Contrapositive: Still $x>0$.  We suppose $x\leq 2$ and show that $x-\frac 2 x \leq 1$.  Since $x>0$ then $x+1>0$.  Therefore each of the following inequations implies the next

\begin{align*}
    x \leq 2 \\
    x-2\leq 0 \\
    (x-2)(x+1) \leq 0 \\
    x^2  - x - 2 \leq 0 \\
    x^2 - 2 \leq x \\
    x - \frac 2 x \leq 1\\
\end{align*}

So $x-\frac 2 x \leq 1$ which is what we wanted. \qed

(c) Contradiction: For contradiction suppose both $x - \frac 2 x > 1$ and that $x\leq 2$.  By the argument in part (b) we already know that $x\leq 2$ entails $x - \frac 2 x \leq 1$. This contradicts our first assumption.  \Lightning\qed
	
\item Suppose $x$ is an integer. Prove that $5x - 7$ is odd if and only if $9x + 2$ is even. \emph{Hint: In both directions, first prove that $x$ is even.}

{\it Solution:} We first prove that if $5x - 7$ is odd then $9x+2$ is even, so assume $5x-7 = 2k+1$ for some integer $k$.  We will first prove that $x$ is even by contradiction, so suppose $x$ is odd and $x = 2m+1$ for an integer $m$.  Then 

\begin{align*}
    5x-7 &= 2k +1\\
    5(2m+1)-7 &= 2k+1 \\
    2m-2 &= 2k+1 \\
    2(m-k) &= 3 \\
\end{align*}

3 is an odd number and by the result above, it is also even, a contradiction.  \Lightning

We have now shown that $x$ is even, so let $x = 2n$.  Then 

\begin{align*}
    9x+2 &= 9(2n)+2\\
    &= 2(9n+1)
\end{align*}

Since $9n+1$ is an integer then the above shows $9x+2$ is even, as desired. 

Now we prove the converse, so assume that $9x+2$ is even and let $9x+2 = 2a$.  We now prove $x$ is even, again by contradiction, so suppose $x$ is odd.  Then let $x = 2b+1$.  Then 

\begin{align*}
    9x+2 &= 2a \\
    9(2b+1) &= 2a \\
    18b+9 &= 2a \\
    9 &= 2(a-9b)
\end{align*}

Since $a-9b$ is an integer the above shows that 9 is even, but also 9 is odd, a contradiction.  \Lightning

Hence we have that $x$ is even so again suppose $x=2n$.  Then 

\begin{align*}
    5x-7 &= 5(2n)-7 \\
    &= 2(5n-4)+1
\end{align*}

and since $5n-4$ is an integer then $5x-7$ is odd. \qed

\item Prove that if $k$ is an odd integer, then the equation $x^2 + x - k = 0$ has no integral solution.

{\it Solution:} Suppose $k$ is an odd integer, and we seek to prove $x^2+x-k=0$ has no integer solutions.  For contradiction suppose $x\in \mathbb Z$ is any integer solution of $x^2+x-k$.  Then $x^2 - x = k$. Since $k$ is odd therefore $x^2+x$ is.  But since $x^2-x=x(x-1)$ then it suffices to show that either $x$ or $x-1$ is even (because an even times any integer is even).  Now to show that, we can show that if $x$ is not even then $x-1$ is.  To that end let's suppose $x$ is odd and $x=2k+1$.  Then $x-1=2k$ is even, as desired.

At this point we have seen that $x(x-1)$ is even, and therefore $k$ is.  But also $k$ is odd, a contradiction.  \Lightning \qed
	
\item Let $m$ and $n$ be integers. Prove that $(m+1)n^2$ is even if and only if $m$ is odd or $n$ is even.

{\it Solution:} First we prove that if $(m+1)n^2$ is even then either $m$ is odd or $n$ is even.  So suppose $(m+1)n^2$ is even and $(m+1)n^2 = 2k$.  To show the disjunction, we prove that if $m$ is not odd then $n$ is even. So suppose $m$ is even and $m=2a$.  Then 

\begin{align*}
    (m+1)n^2 &= 2k \\
    (2a+1)n^2 &= 2k \\
    2an^2 + n^2 &= 2k \\
    n^2 &= 2(k-an^2)\\
\end{align*}

The above shows that $n^2$ is even.  From theorem 1.12 we know that therefore $n$ is even, as desired.  

For the converse, we will show that if $m$ is odd or $n$ is even, then $(m+1)n^2$ is even.  So suppose that either $m$ is odd or $n$ is even.  For a proof by cases first assume $m$ is odd and so $m = 2b+1$.  Then 

\begin{align*}
    (m+1)n^2 & = (2b+2)n^2 \\
    &= 2( [b+1]n^2)\\
\end{align*}

and since $(b+1)n^2$ is an integer the above then shows that $(m+1)n^2$ is even.

For the other case, suppose $n$ is even.  By theorem 1.12 then $n^2$ is even so let $n^2 = 2c$. Then 

\begin{align*}
    (m+1)n^2 &= (m+1)2c = 2([m+1]c)
\end{align*}

which shows $(m+1)n^2$ as desired.  \qed

\item \begin{enumerate}
	\item Let $n$ be an integer. Prove that if $n^2$ is even, then $n^2$ is divisible by 4.
	\item Prove that if $k$ is an odd integer, then $2k$ is not divisible by 4.
	\item Prove that the sum of the squares of two odd integers can not be equal to the square of an integer.
\end{enumerate}

{\it Solution:} (a) Suppose $n^2$ is even and we try to show that $n^2$ is divisible by 4.  If $n^2$ is even then we've already seen that $n$ must be even by theorem 1.12.  So there is some $t$ such that $n = 2t$.  Then $n^2 = 4t^2$ and hence $n^2 = 4s$ where $s$ is the integer $t^2$.  By definition $n^2$ is divisible by 4.

(b) We prove this by the contrapositive, so suppose $2k$ is divisible by 4 and we will try to show that $k$ is not odd.  There exists a $t$ such that $2k = 4t$.  But then $k=2t$ and this directly shows $k$ is even, and hence not odd.

(c) Suppose $a$ and $b$ are odd integer, and we will show that there is no integer $c$ such that $a^2+b^2=c^2$.  Since $a$ and $b$ are odd let $a=2x+1$ and $b=2y+1$.  For contradiction suppose there is an integer $c$ such that $a^2+b^2=c^2$.  Then 

\begin{align*}
    (2x+1)^2+(2y+1)^2 &= c^2\\
    4x^2+4x+1+4y^2+4y+1 & = c^2 \\
    2(2x^2+2x+2y^2+2y+2) &= c^2 \\
\end{align*}

The above shows $c^2$ is even.  By part (a) $c$ is divisible by 4.  Moreover 

\begin{align*}
    2(2[x^2+x+y^2+y+1]) &= c^2 \\
\end{align*}

and $2(x^2+x+y^2+y+1)$ is odd.  If we call $k=2(x^2+x+y^2+y+1)$ then we have $2k = c^2$ with $k$ odd.  Hence by part (b) $c$ is not divisible by 4.  But this contradicts the earlier finding that $c^2$ is divisible by 4. \Lightning


\end{enumerate}
\end{document}