\documentclass{exam}
\usepackage{amsthm,amsmath,amssymb,fullpage}
\usepackage{marvosym}
\usepackage{graphicx}

\DeclareMathAlphabet{\mymathbb}{U}{BOONDOX-ds}{m}{n}

\begin{document}
\noindent \textbf{Name: Adam Frank}\hfill \textbf{\today}

\vspace{.3cm}
\hrule
\begin{center}
{\bf \Large{Math 637: Exam 1}}
\end{center}
\hrule
\vspace{.3cm}


\begin{questions}
\question()\\
Let $G$ be a group.  Prove that if $a\in G$ is the only element of order 2 then $a\in Z(G)$.

\begin{proof}
  Let $b\in G$, then

\begin{align*}
  (bab^{-1})^2 &= bab^{-1}bab^{-1} \\\\
  &= ba^2b^{-1} \\\\
  &= bb^{-1} \\\\
  &= \mymathbb 1
\end{align*}

Therefore $bab^{-1}=a$ or $bab^{-1}=\mymathbb 1$.  The latter implies $a=\mymathbb 1$ so we must have $bab^{-1}=a$ and therefore $ba=ab$ so $a\in Z(G)$.

\end{proof}

\vspace{1cm}

\question()\\
Let $|a|$ and show that $C_G(a)=C_G(a^3)$.

\begin{proof}
  One direction is trivial:  If $b\in C_G(a)$ then $ba^3 = a^3b$ by three operations of ``swapparoo''.

  Now suppose $b\in C_G(a^3)$ so that $ba^3=a^3b$.  Then since $|a|=5$ we have $b=a^3ba^2$.  From this we can derive

  \begin{align*}
    ab &= a(a^3ba^2)\\\\
    &=a^4(a^3ba^2)a^2 \\\\
    &=a^2ba^4 \\\\
    &= a^2(a^3ba^2)a^4 \\\\
    &= ab
  \end{align*}
\end{proof}

\vspace{1cm}

\question()\\
Prove that no group can have exactly two elements of order 2.

\begin{proof}
  Suppose $a\ne b\in G$ each have order 2.  Then we show that also $ab$ has order 2.  For $(ab)^2 = abab = \mymathbb 1$ if and only if $ab = a^{-1}b^{-1}=ab$.
\end{proof}

\vspace{1cm}

\question()\\
Let $p$ be a prime and $Z=\{z\in\mathbb C|z^{p^n}=1 \text{ for some } n\in\mathbb Z^+\}$.  For each $k\in\mathbb Z^+$ define $H_k=\{z\in\mathbb Z|z^{p^k}=1\}$.  Prove that $H_k\leq H_m$ if and only if $k\leq m$.

\begin{proof}
  Suppose $H_k\leq H_m$.  First note that $e^{2i\pi/p^k}\in H_k$ and if $z\in H_k$ then there exists a positive integer $x$ such that $(e^{2i\pi/p^k})^x=z$. Likewise $e^{2i\pi/p^m}\in H_m$ and if $z\in H_m$ then there exists a positive integer $x$ such that $(e^{2i\pi/p^m})^x=z$.  Now by the subgroup relation $e^{2i\pi/p^k}\in H_m$ and therefore there is some positive integer $x$ such that

  \begin{align*}
    e^{2i\pi/p^k} &= (e^{2i\pi/p^m})^x \\\\
    &= e^{2i\pi x/p^m}
  \end{align*}

  Since these complex numbers in polar form are expressed with arguments each less than $2\pi$ we can infer that

  \begin{align*}
    \frac{1}{p^k} = \frac{x}{p^m}
  \end{align*}

  and therefore $p^{m-k}$ is a positive integer, hence $m-k\geq 0$ and so $m\geq k$.

  Conversely, suppose that $m\geq k$.  Certainly the subset relation holds since $(e^{2i\pi/p^k})^{p^m}=(e^{2i\pi})^{p^{m-k}}$ and since $m-k$ is a non-negative integer, we have $(e^{2i\pi})^{p^{m-k}}=1$.  Now let $a,b\in H_k$ so that there exist poisitive integers $x,y$ such that $a=e^{2i\pi x/p^k}$ and $b=e^{2i\pi y/p^k}$.  Then since

  \begin{align*}
    ab^{-1} &= e^{2i\pi(x-y)/p^k}
  \end{align*}

  and since $x-y$ is an integer, then $(ab^{-1})^{p^k}=1$.  Hence $ab^{-1}\in H_k$ and therefore $H_k$ is a subgroup of $H_m$.
\end{proof}

\vspace{1cm}

\question() \\
Let $Z_n$ be a cyclic group of order $n$ and for each $a\in\mathbb Z$ let

\begin{align*}
  \sigma_a: Z_n\rightarrow Z_n
\end{align*}

by $\sigma_a(x)=x^a$.  Prove that $\sigma_a$ is an automorphism if and only if $(a,n)=1$.

\begin{proof}
  Suppose $\sigma_a$ is an automorphism and let $z$ generate the group $Z_n$.  Then $\sigma_a(z)$ must also have order $n$.  Hence $n$ is the minimal natural number such that $z^{an}=\mymathbb 1$.
\end{proof}

\end{questions}



















\end{document}
